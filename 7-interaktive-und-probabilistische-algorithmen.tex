%!TEX root = 0-main.tex

% Author: Philipp Moers <soziflip@gmail.com>



\datum{07.01.16}


\chapter{Probabilistische Algorithmen} % (fold)
\label{cha:probabilistische_algorithmen}



\section{Miller-Rabin Primzahltest Einführung}

Gegeben: $n \in \N$ binär kodiert, $\text{Länge}(n) = \log(n)$

Gefragt: Ist $n$ prim?

Man müsste testen bis $\sqrt{n}$
Anzahl der Zahlen $< \sqrt{n}$: $\sqrt{2^{\log(n)}} = 2^{\frac{1}{2} \log(n)}$, d.\,h. exponentiell in $\text{Länge}(n)$.
Das ist nicht effizient, insbesondere $\notin \ComplexityClassP$.
% todo wo ist das P


\begin{lemma}

    Sei $G$ abelsche Gruppe (endlich) und $U$ eine Untergruppe von $G$.

    Dann ist $|U|$ ein Teiler von $|G|$.

\end{lemma}

\begin{korollar}

    Falls $ggt(x, n)$ = 1, d.\,h. $x \in \Phi(n)$, dann $x^{\phi(n)} = 1 \mod n$,
    wobei $\phi(n) = |\Phi(n)|$ die Anzahl der zu $n$ teilerfremden Zahlen kleiner $n$ ist.

\end{korollar}

\begin{korollar}

    Falls $n$ prim ist, so gilt $x^{n-1} = 1 \mod n$

\end{korollar}


\begin{lemma}

    Falls $x \notin \Phi(n)$, so ist $x^{n-1} \neq 1 \mod n$.

\end{lemma}
\begin{beweis}

    Sei $b = ggt(x, n) > 1$, $v = x^{n-2} \mod n$.

    $x^{n-1} = v * x + u * n $ für ein $u \in \Z$

    $b | RHS \Rightarrow x^{n-1} \neq 1$

\end{beweis}



\section{Fermat-Primzahltest}

Gegeben: $n \in \N$

Wähle zufällig $x \in \{1 \dots n-1\}$.
\\
Bilde $a := x^{n-1} \mod n$ mittels ``Repeated Squaring''.
\\
Falls $a = 1$, so antworte ``ist prim''.
\\
Falls $a \neq 1$, so antworte ``ist zusammengesetzt''.

Ist $n$ tatsächlich prim, so wird mit Sicherheit geantwortet ``ist prim'' (siehe Korollar).
\\
Ist $n$ zusammengesetzt und $x \notin \Phi(n)$, dann wird korrekt geantwortet ``ist zusammengesetzt''.
\\
Wenn $x \in \Phi(n)$, so kann trotzdem $a = 1$ sein, also korrekt geantwortet werden.

Falls $U \neq \Phi(n)$, dann ist $|U|$ ein echter Teiler von $phi(n)$, also $\leq \frac{1}{2} n$.
D.\,h. für mehr als 50\% der $x$ ist dann $a \neq 1$.

Falls aber für alle $x \in \Phi(n)$ gilt $x^{n-1} = 1 \mod n$ und doch $n$ zusammengesetzt ist, dann funktioniert der Fermat-Test nicht gut. Solche $n$ heißen Carmichaelzahl, falls $n$ nicht prim, aber $x^{n-1} = 1 \mod n \forall x \in \Phi(n)$. Solche Carmichaelzahlen sind sehr selten.



\begin{satz}

    Ist $n$ Carmichaelzahl, so ist der Anteil der $x \in \{1 \dots n-1\}$, sodass $\exists i: ggt(x^{\frac{n-1}{2^i}} - 1, n) > 1$ größer oder gleich 75\%.

\end{satz}



\section{Miller-Rabin Primzahltest}

Gegeben $n \in \N$

Wähle zufällig $x \in \{1 \dots n-1\}$.
\\
Bilde $a = x^{n-1} \mod n$.
\\
Falls $a \neq 1$, so ist $n$ sicher zusammengesetzt.
\\
Falls $a = 1$, berechne $ggt(x^{\frac{n-1}{2^i}} - 1, n)$ für $i = 1 \dots \log(n)$. Falls einmal $> 1$, so ist $n$ sicher zusammengesetzt. Sonst antworte ``ist prim''.

Fehlerwahrscheinlichkeit $\leq 50\%$.





\section{Testen von polynomiellen Identitäten}

% TODO vbl?
Gegeben: zwei Polynome $p(x), q(x)$ mit Koeffizienten aus $\Z$ und $n$ Vbl $x = x_1 \dots x_n$.

Gefragt: Ist $p(x) = q(x)$?


\begin{beispiel}

    $p = (a^2 + b^2 + c^2 + d^2)(A^2 + B^2 + C^2 + D^2)$

    $q = (aA + bB + cC + dD)^2 + \\
         (aB + bA + cD + dC)^2 + \\
         (aC + bD + cA + dB)^2 + \\
         (aD + bC + cB + dA)^2 \\
    $

    Hier ist in der Tat $p = q$.

\end{beispiel}

Für diese Aufgabe ist kein Polynomialzeitverfahren bekannt.
Ausmultiplizieren hat im Allgemeinen exponentielle Laufzeit, falls $*$ und $+$ geschachtelt auftreten.



\textbf{Probabilistisches Verfahren:}

Wähle zufällig $x \in \{-nd, \dots 0, 1, 2, \dots , nd\}^n$,
wobei $n = \text{ Anzahl Vbl } $ und $  d \geq Grad(p), Grad(q)$.

Wir zeigen jetzt, dass die Fehlerwahrscheinlichkeit kleiner oder gleich 50\% ist.

\begin{lemma}

    \definiere{Lemma von Schwarz-Zippe}

    Sei $p$ ein Polynom vom Grad $d$ in $n$ Variablen und $p \neq q$.

    Dann hat $p$ im Bereich $\{-N, \dots N\}^n$ höchstens $dn(2N+1)^{n-1}$ Nullstellen.

\end{lemma}

Falls $N = nd$, dann ist $\frac{(2N+1)^{n-1} * dn}{(2N+1)^n}  =  \frac{dn}{2N+1} = \frac{dn}{2dn+1} \leq 50\%$.








\datum{11.01.16}


\begin{definition}

    Ein \definiere{Perfektes Matching} zu Graph $G= (V, E)$ ist eine
    Kantenauswahl $T \subseteq E$, sodass
    $u, v \in T, u \neq v \rightarrow u \cap v = \emptyset$.
    (Die ausgewählten Kanten haben keine Punkte gemeinsam.)

    Ein Graph ist perfekt, wenn $\bigcup T = V$.
    (Alle Knoten werden gematcht.)

\end{definition}


\begin{satz}
    Sei $G=(V,E)$ ein Graph und $A$ die zugehörige Adjazenzmatrix.
    $G$ hat ein perfektes Matching genau dann, wenn $det(A) \neq 0$.
\end{satz}



\begin{definition}
    Eine \definiere{Probabilistische Turingmaschine} ist wie eine nichtdeterministische
    Turingmaschine aufgebaut. Zu jedem Paar $q, a$ gibt es exakt zwei Quintupel
    (ggf.\ künstlich aufblähen).
\end{definition}

Deutung:
Passendes Quintupel wird jeweils mit Wahrscheinlichkeit 50\% ausgewählt.
Die Wahrscheinlichkeit, dass $T$ die Eingabe $x$ akzeptiert, ist
$\frac{\text{Anzahl akzeptierender Berechnungen}}{\text{Anzahl aller Berechnungen}}$.

Polynomialzeitbeschränkt, falls Laufzeit $\leq p(|x|)$ unabhängig von den Zufallsentscheidungen.


\begin{definition}
    Eine Sprache $X \subseteq \Sigma^\ast$ ist in der Komplexitätsklasse $\ComplexityClassPP$,
    wenn eine Probabilistische Turingmaschine $T$ existiert, die
    \begin{itemize}
        \item polynomiell zeitbeschränkt ist,
        \item $x \in X \Leftrightarrow \text{Akzeptanzwahrscheinlichkeit} \geq 50\%$.
    \end{itemize}
\end{definition}

\begin{satz}
    $\ComplexityClassP \subseteq PSPACE$
\end{satz}
\begin{beweis}
    Alle Berechnungen durchprobieren (Platz jeweils wiederverwenden) und Anzahl der
    akzeptierenden Berechnungen mitzählen.
\end{beweis}

\begin{satz}
    $\ComplexityClassNP \subseteq PSPACE$
\end{satz}
\begin{beweis}
    Sei eine nichtdeterministische Turingmaschine $T$ für $X$ gegeben
    (O.B.d.A. im 2. Nachfolgequintupelformat).
    Zunächst mit 50\% Wahrscheinlichkeit einfach so akzeptieren,
    anschließend $T$ als Probabilistische Turingmaschine fahren.
    \\
    $x \in X$: Wahrscheinlichkeit der Akzeptanz $\geq 50\%$
    \\
    $x \notin X$: Wahrscheinlichkeit der Akzeptanz $= 50\%$
\end{beweis}

\begin{satz}
    (Beispiel Spielman et al.)

    $\ComplexityClassPP$ ist unter $\cap, \cup$ abgeschlossen.
\end{satz}
Der Beweis dazu ist kompliziert.



\begin{definition}
    Komplexitätsklassen $\ComplexityClassR, \ComplexityClassRP$

    $X \in \ComplexityClassR$ genau dann, wenn eine polynomiell zeitbeschränkte Probabilistische Turingmaschine $T$ existiert mit
    \begin{itemize}
        \item $x \in X \Rightarrow$ Wahrscheinlichkeit der Akzeptanz $\geq 50\%$
        \item $x \notin X \Rightarrow$ Wahrscheinlichkeit der Akzeptanz $= 0$
    \end{itemize}
\end{definition}

\begin{beispiel}
    $PRIM \in \ComplexityClassR$
\end{beispiel}


\begin{definition}
    Komplexitätsklasse $\ComplexityClassBPP$

    $X \in \ComplexityClassBPP$ genau dann, wenn eine polynomiell zeitbeschränkte Probabilistische Turingmaschine $T$ existiert mit
    \begin{itemize}
        \item $x \in X $: Wahrscheinlichkeit der Akzeptanz $\geq 75\%$
        \item $x \notin X $: Wahrscheinlichkeit der Akzeptanz $\leq 75\%$
    \end{itemize}
\end{definition}

\begin{satz}
    Sei $A \in \ComplexityClassBPP$ und $q(n)$ ein Polynom.
    \\
    Es existiert eine Probabilistische Turingmaschine $T$ für $A$
    mit Fehlerwahrscheinlichkeit $\leq e^{-q(|x|)}$ für Eingabe $x$.
\end{satz}
\begin{beweis}
    Beweisidee: $A$ auf Eingabe $x$ wiederholt, d.\,h. $t = t(|x|)$ mal ablaufen lassen.
    Am Ende diejenige Antwort, die häufiger gegeben wurde, nach außen reichen.

    $x \in A$ Anzahl der Akzeptanzen binomialverteilt mit Parameter $p \geq 75\%$.
    Sei $S$ diese Anzahl (Zufallsvariable).
    Wir möchten $Pr(S \leq \frac{t}{2}) \leq e^{-q(|x|)}$.

    Es gilt allgemein die Chernoff-Schranke.
    (\textit{siehe Details im Buch})

\end{beweis}

\datum{14.1.2016}        

\section{Chernoff Schranke}
Seien $X_1 \dots X_n$ $0-1$
Zufallsvariablen mit $Pr(X_i=1) = P_i$

Setze $S=X_1 + \dots + X_n$ \\
$\mu = E[S] = p_1 \dots  + p_n$

Für jedes $S > 0$

$PR (S>(1+S)) \mu \leq (\frac{e^\delta}{(1+\delta)^{+\delta}})^\delta$
Falls zusätzlich $\delta < 1$ \\
$Pr(S\leq (1- \delta) \mu ) \leq e^{\frac{- \delta^2}{2} * \mu}$

In der Literatur gibt es andere Versionen.

\paragraph{Anmerkung}
t-fache unabhängige Ausführung eines BRP-Algorithmus für
$A\subseteq \Sigma^*$ bei Eingabe x. $x \in A$: $\mu = 0,75 * t$

Fehlerwahrscheinlichkeit:
$Pr(S\leq \frac{t}{2}) = Pr(S \leq (1 - \delta) \mu)$
Für $\delta = \frac{2}{3}$

t ist also so zu wählen, dass

$e ^ {\frac{\frac{-2}{3}^2}{2} * \frac{3}{4} * t} \leq e^{-q(|x|)}$

Also $q(|x|) \geq \frac{\frac{3}{2}^2}{2} * \frac{3}{x} * t$

$q(|x|) \geq 24 t$

$t \notin A : \mu = \frac{1}{4} * t $

$ PR( S\geq \frac{t}{s} = PR (S \grq (1+1) \mu ) $
$ \geq \frac{e}{4}^\frac{t}{4} = (e^{-0.386})^\frac{t}{4}$

Es muss also
$0.386 * \frac{t}{4} \leq q(|x|)$
Wahr, wenn $t(|x|)=24 * g(|x|)$

\subsection{Anwendung}
Satz: $BPP \subseteq \frac{P}{Poly}$
$\frac{P}{Poly} = \{A | \forall n \exists$  Schaltkreis $C. |C| \leq p(n) \forall x . |x| < n : x \in A \Leftrightarrow C(x) = 1\} $

$SAT \in \frac{P}{Poly} \Rightarrow PH$ kollabiert

\begin{beweis}
Sei ein BPP Algirithmus M für A gegeben. OBdA(vorhergehender Satz) können wir annehmen, dass

$ x \in A  \Rightarrow Pr (M akzeptiert . x) \geq 1-2^{-(|x| + 1)}$

$ x \notin  A  \Rightarrow Pr (M akzeptiert . x) \leq 1-2^{-(|x| + 1)}$

Wir nehmen außerdem an, dass $| \Sigma  | = 2$ (sonst ersetze 2 durch $| \Sigma|$).

Anteil der Zufallsentscheidungen, der für ein bestimmtes $x\in \Sigma^*$
ungünstig ist, ist $\leq 2 ^{-(|x| + 1)}$, d.h. $\leq 2^{-(n+1)}$ falls
$|x| = n$

Der Anteil der Zufallsentscheidungen, der ffür mindestens ein $x\in \Sigma^*$
ungünstig ist, also $\leq 2^n * 2^{-(n+1)} \leq \frac{1}{2}$

Der Anteil der Zufallsentscheidungen die für jedes x mit $|x|$ = n günstig  sind,
ist $>0$, ja sogar $> \frac{1}{2}$. Der Schaltkreis $C$, der für alle Inputs der
Größe $n$ funktioniert ergibt sich aus $M$ in dem eine dieser universell
günstigen Entscheidungen festverdrahtet wird.
\end{beweis}

BPP, RP, co-RP \checked

ZPP : Antwort immer richtig mit $W\leq 50$ darf "dont know'' geantwortet werden. Laufzeit in jedem Fall polynomiell.

Offensichtlich $ZPP = RP \cap co - RP$

Las Vegas Algorithmus (verallgemeinert ZPP) Liefert nur richtige Ergebnisse aber manchmal($W<50\%$) kein Ergebnis ("dont know'')

Monte Carlo Algorthmus Liefert Ergenbis, dass mit $W \leq 25\% falsch ist (verallgemeinert BPP)$

\begin{beispiel}
für Las Vegas

Randomisiertes Quicksort (Laufzeitschranke $n log n$ mit laufen lassen
"dont know'' falls nicht richtig)
\end{beispiel}

Allg. ist ein Las Vegas Alg. ersetzbar durch einen, der immer richtig
antwortet, aber mit einer gewissen Wahrscheinlichkeit nicht
terminiert. Letzteres mit Wahrscheinlichkeit $0$.

(Wiederholung)
Folgene formale Version:
$ZPP = \{A |$ es existiert ein Entscheidungsverfahren, für A dessen Laufzeit eine Zufallsvariable ist mit polynomiellem Erwartungswert $\}$. beschränkt.      

\section{Interaktive Beweissysteme} 
\begin{itemize}
\item Polynomiell laufzeitbeschränkter Verifizierer (``wir''): $V$
\item Mit umbeschränkter Rechenkraft ausgestatteter Beweiser (``sie''): $P$
\end{itemize}

% Jetzt kommt eine GRafik
%Eingabeband

$P$ und $V$ haben spezielle Zustände zur Beendigung von Phasen.

Bei gegebnere Eingabe $x$ auf dem Eingabend finden eine bestimmte Zahl von Runden statt.
$Zahl \leq poly (|x|)$
Zu Beginn jeder Runde schreitet $V$ ein Wort aufs Frageband. Anschließend schreibt $P$ eine Antwort aufs Antwortband.

Zum Schluss(alle Runden vorbei) entscheidet ob akzeptiert oder nicht.

\begin{definition}
Sei $X \subseteq \Sigma^*$. Eine polynomiell beschränkte Turingmaschine $V$ ist polynomiell verifizierbar für $x$
genau dann wenn $ \exists$ Prover $P$
$\forall x \in A$ . $V \leftrightarrow P$ auf $x$ akzeptiert. $\forall x \notin A$ . $V \leftrightarrow P$   akzeptiert auf $x$ nicht.

Falls Verifizierer deterministische polynomiell beschränkt
Turingmaschine sind, so spricht man von der Klasse DIP (falls Problem für
die ein deterministischer polynomiellialzeit Verifizierer existiert.)
\end{definition}

\begin{satz}
DIP = NP

\begin{description}        
\item["$\subseteq$''] Verwendet Formulierung von NP mit Polynom großen "Lösungen''.
\item["$\supseteq$''] Sei $V$ ein DIP Verifizierer für $X$ und P der
  zugehörigem Prover. Komplettes Transcript raten bzw. als Lösung
  verwenden.
\end{description}
\end{satz}      

\begin{definition}
$ \subseteq \Sigma^*$ ist in $TP$
falls eine probailistischer polynomialzeit beschrkänkter Verifizierer $V$
existiert so, dass $\exists$ Prover $P$

$ x\in A : Pr(V \leftrightarrow P$ auf $x$ akzeptiert) $\geq 75\%$.

$ x \notin A : \forall$ Prover $P$.

$P_1 (V \leftrightarrow P$ auf $x$ akzeptiert) $\leq 25\%$.

Evidenz für IP ``$\supsetneqq$'' $NP$.

Sei $G$ das Problem Graph-Isomorphismus. Offensichtligh $GI \in NP$ Isomorphismus vorlgegen.

Man weiß nicht ob $\overline{GI} \in NP$.

Aber: $\overline{GI} \in IP$.

$V:$

\begin{itemize}
\item Gegenben: $G$,  $H$ Graphen
\item Wähle zufällig (1:1) $b\in \{0,1\}$
\item if b then K:= G else K:=H
\item Wähle zufällige Permutation $\pi$ auf den Knoten von $K$ und stelle an $P$ die Fragen $\pi(K)$ $P$ antwortet mit $b\in \{0,1\}$
Akzeptiere, falls $b=b'$
\end{itemize}

Wiederhole dies 2x (akzeptiert ``wirklich'' nur falls beide Male akzeptiert wurde).

Dies ist korrektes Protokoll für $\overline{GI}$.

\end{definition}

% chapter probabilistische_algorithmen (end)

%%% Local Variables:
%%% mode: latex
%%% TeX-master: "0-main"
%%% ispell-local-dictionary: "de-neu"
%%% eval: (highlight-parentheses-mode)
%%% End:
