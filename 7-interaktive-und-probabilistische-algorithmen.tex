%!TEX root = 0-main.tex

% Author: Philipp Moers <soziflip@gmail.com>



\datum{07.01.16}


\chapter{Probabilistische Algorithmen} % (fold)
\label{cha:probabilistische_algorithmen}



\section{Miller-Rabin Primzahltest Einführung}

Gegeben: $n \in \N$ binär kodiert, $\text{Länge}(n) = \log(n)$

Gefragt: Ist $n$ prim?

Man müsste testen bis $\sqrt{n}$
Anzahl der Zahlen $< \sqrt{n}$: $\sqrt{2^{\log(n)}} = 2^{\frac{1}{2} \log(n)}$, d.\,h. exponentiell in $\text{Länge}(n)$.
Das ist nicht effizient, insbesondere $\notin \ComplexityClassP$.
% todo wo ist das P


\begin{lemma}

    Sei $G$ abelsche Gruppe (endlich) und $U$ eine Untergruppe von $G$.

    Dann ist $|U|$ ein Teiler von $|G|$.

\end{lemma}

\begin{korollar}

    Falls $ggt(x, n)$ = 1, d.\,h. $x \in \Phi(n)$, dann $x^{\phi(n)} = 1 \mod n$,
    wobei $\phi(n) = |\Phi(n)|$ die Anzahl der zu $n$ teilerfremden Zahlen kleiner $n$ ist.

\end{korollar}

\begin{korollar}

    Falls $n$ prim ist, so gilt $x^{n-1} = 1 \mod n$

\end{korollar}


\begin{lemma}

    Falls $x \notin \Phi(n)$, so ist $x^{n-1} \neq 1 \mod n$.

\end{lemma}
\begin{beweis}

    Sei $b = ggt(x, n) > 1$, $v = x^{n-2} \mod n$.

    $x^{n-1} = v * x + u * n $ für ein $u \in \Z$

    $b | RHS \Rightarrow x^{n-1} \neq 1$

\end{beweis}



\section{Fermat-Primzahltest}

Gegeben: $n \in \N$

Wähle zufällig $x \in \{1 \dots n-1\}$.
\\
Bilde $a := x^{n-1} \mod n$ mittels ``Repeated Squaring''.
\\
Falls $a = 1$, so antworte ``ist prim''.
\\
Falls $a \neq 1$, so antworte ``ist zusammengesetzt''.

Ist $n$ tatsächlich prim, so wird mit Sicherheit geantwortet ``ist prim'' (siehe Korollar).
\\
Ist $n$ zusammengesetzt und $x \notin \Phi(n)$, dann wird korrekt geantwortet ``ist zusammengesetzt''.
\\
Wenn $x \in \Phi(n)$, so kann trotzdem $a = 1$ sein, also korrekt geantwortet werden.

Falls $U \neq \Phi(n)$, dann ist $|U|$ ein echter Teiler von $phi(n)$, also $\leq \frac{1}{2} n$.
D.\,h. für mehr als 50\% der $x$ ist dann $a \neq 1$.

Falls aber für alle $x \in \Phi(n)$ gilt $x^{n-1} = 1 \mod n$ und doch $n$ zusammengesetzt ist, dann funktioniert der Fermat-Test nicht gut. Solche $n$ heißen Carmichaelzahl, falls $n$ nicht prim, aber $x^{n-1} = 1 \mod n \forall x \in \Phi(n)$. Solche Carmichaelzahlen sind sehr selten.



\begin{satz}

    Ist $n$ Carmichaelzahl, so ist der Anteil der $x \in \{1 \dots n-1\}$, sodass $\exists i: ggt(x^{\frac{n-1}{2^i}} - 1, n) > 1$ größer oder gleich 75\%.

\end{satz}



\section{Miller-Rabin Primzahltest}

Gegeben $n \in \N$

Wähle zufällig $x \in \{1 \dots n-1\}$.
\\
Bilde $a = x^{n-1} \mod n$.
\\
Falls $a \neq 1$, so ist $n$ sicher zusammengesetzt.
\\
Falls $a = 1$, berechne $ggt(x^{\frac{n-1}{2^i}} - 1, n)$ für $i = 1 \dots \log(n)$. Falls einmal $> 1$, so ist $n$ sicher zusammengesetzt. Sonst antworte ``ist prim''.

Fehlerwahrscheinlichkeit $\leq 50\%$.





\section{Testen von polynomiellen Identitäten}

% TODO vbl?
Gegeben: zwei Polynome $p(x), q(x)$ mit Koeffizienten aus $\Z$ und $n$ Vbl $x = x_1 \dots x_n$.

Gefragt: Ist $p(x) = q(x)$?


\begin{beispiel}

    $p = (a^2 + b^2 + c^2 + d^2)(A^2 + B^2 + C^2 + D^2)$

    $q = (aA + bB + cC + dD)^2 + \\
         (aB + bA + cD + dC)^2 + \\
         (aC + bD + cA + dB)^2 + \\
         (aD + bC + cB + dA)^2 \\
    $

    Hier ist in der Tat $p = q$.

\end{beispiel}

Für diese Aufgabe ist kein Polynomialzeitverfahren bekannt.
Ausmultiplizieren hat im Allgemeinen exponentielle Laufzeit, falls $*$ und $+$ geschachtelt auftreten.



\textbf{Probabilistisches Verfahren:}

Wähle zufällig $x \in \{-nd, \dots 0, 1, 2, \dots , nd\}^n$,
wobei $n = \text{ Anzahl Vbl } $ und $  d \geq Grad(p), Grad(q)$.

Wir zeigen jetzt, dass die Fehlerwahrscheinlichkeit kleiner oder gleich 50\% ist.

\begin{lemma}

    \definiere{Lemma von Schwarz-Zippe}

    Sei $p$ ein Polynom vom Grad $d$ in $n$ Variablen und $p \neq q$.

    Dann hat $p$ im Bereich $\{-N, \dots N\}^n$ höchstens $dn(2N+1)^{n-1}$ Nullstellen.

\end{lemma}

Falls $N = nd$, dann ist $\frac{(2N+1)^{n-1} * dn}{(2N+1)^n}  =  \frac{dn}{2N+1} = \frac{dn}{2dn+1} \leq 50\%$.








\datum{11.01.16}


\begin{definition}

    Ein \definiere{Perfektes Matching} zu Graph $G= (V, E)$ ist eine
    Kantenauswahl $T \subseteq E$, sodass
    $u, v \in T, u \neq v \rightarrow u \cap v = \emptyset$.
    (Die ausgewählten Kanten haben keine Punkte gemeinsam.)

    Ein Graph ist perfekt, wenn $\bigcup T = V$.
    (Alle Knoten werden gematcht.)

\end{definition}


\begin{satz}
    Sei $G=(V,E)$ ein Graph und $A$ die zugehörige Adjazenzmatrix.
    $G$ hat ein perfektes Matching genau dann, wenn $det(A) \neq 0$.
\end{satz}



\begin{definition}
    Eine \definiere{Probabilistische Turingmaschine} ist wie eine nichtdeterministische
    Turingmaschine aufgebaut. Zu jedem Paar $q, a$ gibt es exakt zwei Quintupel
    (ggf.\ künstlich aufblähen).
\end{definition}

Deutung:
Passendes Quintupel wird jeweils mit Wahrscheinlichkeit 50\% ausgewählt.
Die Wahrscheinlichkeit, dass $T$ die Eingabe $x$ akzeptiert, ist
$\frac{\text{Anzahl akzeptierender Berechnungen}}{\text{Anzahl aller Berechnungen}}$.

Polynomialzeitbeschränkt, falls Laufzeit $\leq p(|x|)$ unabhängig von den Zufallsentscheidungen.


\begin{definition}
    Eine Sprache $X \subseteq \Sigma^\ast$ ist in der Komplexitätsklasse $\ComplexityClassPP$,
    wenn eine Probabilistische Turingmaschine $T$ existiert, die
    \begin{itemize}
        \item polynomiell zeitbeschränkt ist,
        \item $x \in X \Leftrightarrow \text{Akzeptanzwahrscheinlichkeit} \geq 50\%$.
    \end{itemize}
\end{definition}

\begin{satz}
    $\ComplexityClassP \subseteq PSPACE$
\end{satz}
\begin{beweis}
    Alle Berechnungen durchprobieren (Platz jeweils wiederverwenden) und Anzahl der
    akzeptierenden Berechnungen mitzählen.
\end{beweis}

\begin{satz}
    $\ComplexityClassNP \subseteq PSPACE$
\end{satz}
\begin{beweis}
    Sei eine nichtdeterministische Turingmaschine $T$ für $X$ gegeben
    (O.B.d.A. im 2. Nachfolgequintupelformat).
    Zunächst mit 50\% Wahrscheinlichkeit einfach so akzeptieren,
    anschließend $T$ als Probabilistische Turingmaschine fahren.
    \\
    $x \in X$: Wahrscheinlichkeit der Akzeptanz $\geq 50\%$
    \\
    $x \notin X$: Wahrscheinlichkeit der Akzeptanz $= 50\%$
\end{beweis}

\begin{satz}
    (Beispiel Spielman et al.)

    $\ComplexityClassPP$ ist unter $\cap, \cup$ abgeschlossen.
\end{satz}
Der Beweis dazu ist kompliziert.



\begin{definition}
    Komplexitätsklassen $\ComplexityClassR, \ComplexityClassRP$

    $X \in \ComplexityClassR$ genau dann, wenn eine polynomiell zeitbeschränkte Probabilistische Turingmaschine $T$ existiert mit
    \begin{itemize}
        \item $x \in X \Rightarrow$ Wahrscheinlichkeit der Akzeptanz $\geq 50\%$
        \item $x \notin X \Rightarrow$ Wahrscheinlichkeit der Akzeptanz $= 0$
    \end{itemize}
\end{definition}

\begin{beispiel}
    $PRIM \in \ComplexityClassR$
\end{beispiel}


\begin{definition}
    Komplexitätsklasse $\ComplexityClassBPP$

    $X \in \ComplexityClassBPP$ genau dann, wenn eine polynomiell zeitbeschränkte Probabilistische Turingmaschine $T$ existiert mit
    \begin{itemize}
        %\item $x \in X $: Wahrscheinlichkeit der Akzeptanz $\geq 75\%$
        \item $x \in X $: Wahrscheinlichkeit der Akzeptanz $> 75\%$
        \item $x \notin X $: Wahrscheinlichkeit der Akzeptanz $\leq 25\%$
    \end{itemize}
\end{definition}

\begin{satz}
    Sei $A \in \ComplexityClassBPP$ und $q(n)$ ein Polynom.
    \\
    Es existiert eine Probabilistische Turingmaschine $T$ für $A$
    mit Fehlerwahrscheinlichkeit $\leq e^{-q(|x|)}$ für Eingabe $x$.
\end{satz}
\begin{beweis}
    Beweisidee: $A$ auf Eingabe $x$ wiederholt, d.\,h. $t = t(|x|)$ mal ablaufen lassen.
    Am Ende diejenige Antwort, die häufiger gegeben wurde, nach außen reichen.

    $x \in A$ Anzahl der Akzeptanzen binomialverteilt mit Parameter $p \geq 75\%$.
    Sei $S$ diese Anzahl (Zufallsvariable).
    Wir möchten $Pr(S \leq \frac{t}{2}) \leq e^{-q(|x|)}$.

    Es gilt allgemein die Chernoff-Schranke.
    (\textit{siehe Details im Buch})

\end{beweis}











\datum{14.01.2016}

\section{Chernoff Schranke}
Seien $X_1 \dots X_n$ $0-1$
Zufallsvariablen mit $Pr(X_i=1) = P_i$

Setze $S=X_1 + \dots + X_n$ \\
$\mu = E[S] = p_1 \dots  + p_n$

Für jedes $S > 0$

$PR (S>(1+S)) \mu \leq (\frac{e^\delta}{(1+\delta)^{+\delta}})^\delta$
Falls zusätzlich $\delta < 1$ \\
$Pr(S\leq (1- \delta) \mu ) \leq e^{\frac{- \delta^2}{2} * \mu}$

In der Literatur gibt es andere Versionen.

\paragraph{Anmerkung}
t-fache unabhängige Ausführung eines BRP-Algorithmus für
$A\subseteq \Sigma^*$ bei Eingabe x. $x \in A$: $\mu = 0,75 * t$

Fehlerwahrscheinlichkeit:
$Pr(S\leq \frac{t}{2}) = Pr(S \leq (1 - \delta) \mu)$
Für $\delta = \frac{2}{3}$

t ist also so zu wählen, dass

$e ^ {\frac{\frac{-2}{3}^2}{2} * \frac{3}{4} * t} \leq e^{-q(|x|)}$

Also $q(|x|) \geq \frac{\frac{3}{2}^2}{2} * \frac{3}{x} * t$

$q(|x|) \geq 24 t$

$t \notin A : \mu = \frac{1}{4} * t $

$ PR( S\geq \frac{t}{s} = PR (S \grq (1+1) \mu ) $
$ \geq \frac{e}{4}^\frac{t}{4} = (e^{-0.386})^\frac{t}{4}$

Es muss also
$0.386 * \frac{t}{4} \leq q(|x|)$
Wahr, wenn $t(|x|)=24 * g(|x|)$

\subsection{Anwendung}
Satz: $BPP \subseteq \frac{P}{Poly}$
$\frac{P}{Poly} = \{A | \forall n \exists$  Schaltkreis $C. |C| \leq p(n) \forall x . |x| < n : x \in A \Leftrightarrow C(x) = 1\} $

$SAT \in \frac{P}{Poly} \Rightarrow PH$ kollabiert

\begin{beweis}
Sei ein BPP Algirithmus M für A gegeben. OBdA(vorhergehender Satz) können wir annehmen, dass

$ x \in A  \Rightarrow Pr (M akzeptiert . x) \geq 1-2^{-(|x| + 1)}$

$ x \notin  A  \Rightarrow Pr (M akzeptiert . x) \leq 1-2^{-(|x| + 1)}$

Wir nehmen außerdem an, dass $| \Sigma  | = 2$ (sonst ersetze 2 durch $| \Sigma|$).

Anteil der Zufallsentscheidungen, der für ein bestimmtes $x\in \Sigma^*$
ungünstig ist, ist $\leq 2 ^{-(|x| + 1)}$, d.h. $\leq 2^{-(n+1)}$ falls
$|x| = n$

Der Anteil der Zufallsentscheidungen, der ffür mindestens ein $x\in \Sigma^*$
ungünstig ist, also $\leq 2^n * 2^{-(n+1)} \leq \frac{1}{2}$

Der Anteil der Zufallsentscheidungen die für jedes x mit $|x|$ = n günstig  sind,
ist $>0$, ja sogar $> \frac{1}{2}$. Der Schaltkreis $C$, der für alle Inputs der
Größe $n$ funktioniert ergibt sich aus $M$ in dem eine dieser universell
günstigen Entscheidungen festverdrahtet wird.
\end{beweis}

BPP, RP, co-RP \checked

ZPP : Antwort immer richtig mit $W\leq 50$ darf "dont know'' geantwortet werden. Laufzeit in jedem Fall polynomiell.

Offensichtlich $ZPP = RP \cap co - RP$

Las Vegas Algorithmus (verallgemeinert ZPP) Liefert nur richtige Ergebnisse aber manchmal($W<50\%$) kein Ergebnis ("dont know'')

Monte Carlo Algorthmus Liefert Ergenbis, dass mit $W \leq 25\% falsch ist (verallgemeinert BPP)$

\begin{beispiel}
für Las Vegas

Randomisiertes Quicksort (Laufzeitschranke $n log n$ mit laufen lassen
"dont know'' falls nicht richtig)
\end{beispiel}

Allg. ist ein Las Vegas Alg. ersetzbar durch einen, der immer richtig
antwortet, aber mit einer gewissen Wahrscheinlichkeit nicht
terminiert. Letzteres mit Wahrscheinlichkeit $0$.

(Wiederholung)
Folgene formale Version:
$ZPP = \{A |$ es existiert ein Entscheidungsverfahren, für A dessen Laufzeit eine Zufallsvariable ist mit polynomiellem Erwartungswert $\}$. beschränkt.

\section{Interaktive Beweissysteme}
\begin{itemize}
\item Polynomiell laufzeitbeschränkter Verifizierer (``wir''): $V$
\item Mit umbeschränkter Rechenkraft ausgestatteter Beweiser (``sie''): $P$
\end{itemize}

% Jetzt kommt eine GRafik
%Eingabeband

$P$ und $V$ haben spezielle Zustände zur Beendigung von Phasen.

Bei gegebnere Eingabe $x$ auf dem Eingabend finden eine bestimmte Zahl von Runden statt.
$Zahl \leq poly (|x|)$
Zu Beginn jeder Runde schreitet $V$ ein Wort aufs Frageband. Anschließend schreibt $P$ eine Antwort aufs Antwortband.

Zum Schluss(alle Runden vorbei) entscheidet ob akzeptiert oder nicht.

\begin{definition}
Sei $X \subseteq \Sigma^*$. Eine polynomiell beschränkte Turingmaschine $V$ ist polynomiell verifizierbar für $x$
genau dann wenn $ \exists$ Prover $P$
$\forall x \in A$ . $V \leftrightarrow P$ auf $x$ akzeptiert. $\forall x \notin A$ . $V \leftrightarrow P$   akzeptiert auf $x$ nicht.

Falls Verifizierer deterministische polynomiell beschränkt
Turingmaschine sind, so spricht man von der Klasse DIP (falls Problem für
die ein deterministischer polynomiellialzeit Verifizierer existiert.)
\end{definition}

\begin{satz}
DIP = NP

\begin{description}
\item["$\subseteq$''] Verwendet Formulierung von NP mit Polynom großen "Lösungen''.
\item["$\supseteq$''] Sei $V$ ein DIP Verifizierer für $X$ und P der
  zugehörigem Prover. Komplettes Transcript raten bzw. als Lösung
  verwenden.
\end{description}
\end{satz}

\begin{definition}
$ \subseteq \Sigma^*$ ist in $TP$
falls eine probailistischer polynomialzeit beschrkänkter Verifizierer $V$
existiert so, dass $\exists$ Prover $P$

$ x\in A : Pr(V \leftrightarrow P$ auf $x$ akzeptiert) $\geq 75\%$.

$ x \notin A : \forall$ Prover $P$.

$P_1 (V \leftrightarrow P$ auf $x$ akzeptiert) $\leq 25\%$.

Evidenz für IP ``$\supsetneqq$'' $NP$.

Sei $G$ das Problem Graph-Isomorphismus. Offensichtlich $GI \in NP$ Isomorphismus vorlgegen.

Man weiß nicht ob $\overline{GI} \in NP$.

Aber: $\overline{GI} \in IP$.

$V:$

\begin{itemize}
\item Gegenben: $G$,  $H$ Graphen
\item Wähle zufällig (1:1) $b\in \{0,1\}$
\item if b then K:= G else K:=H
\item Wähle zufällige Permutation $\pi$ auf den Knoten von $K$ und stelle an $P$ die Fragen $\pi(K)$ $P$ antwortet mit $b\in \{0,1\}$
Akzeptiere, falls $b=b'$
\end{itemize}

Wiederhole dies 2x (akzeptiert ``wirklich'' nur falls beide Male akzeptiert wurde).

Dies ist korrektes Protokoll für $\overline{GI}$.

\end{definition}









\datum{25.01.2016}


% wiederholung: TODO merge with above

\begin{definition}
    \definiere{Interaktives Beweissysteme}
    \begin{align*}
        L \in \mathit{IP} & \text{ genau dann, wenn }
        \\
        & \exists \text{ polynomialzeitbeschränkte, probabilistische Turingmaschine Verifier } V:
        \\
        & \exists \text{ Prover } P:
        \\
        & x \in L \Rightarrow (P,V) \text{ akzepiert auf } x \text{ mit Wahrscheinlichkeit } \geq 75\%
        \\
        & x \notin L \Rightarrow \forall P: (P,V) \text{ verwirft auf } x \text{ mit Wahrscheinlichkeit } \geq 75\%
    \end{align*}

    Der Prover ist eine deterministische Turingmaschine ohne
    Ressourcenbeschränkung. In jeder Runde schickt der Verifizierer eine Frage
    an den Prover und der Prover eine Antwort an den Verifizierer.
    Die Anzahl der Runden ist polynomiell beschränkt.
    Ausschließlich $V$ entscheidet, ob akzeptiert wird.

\end{definition}




 % so jetzt geht's los


\subsection{Graphisomorphismus}
\label{sec:graphisomorphismus}


Verifizierer für das Problem, ob zwei Graphen \underline{nicht} isomorph sind, $\overline{GI}$:

Gegeben: $G_1 = (V_1, E_1), G_2 = (V_2, E_2)$

Falls $|V_1| \neq |V_2|$, dann verwerfen.

Wähle zufällig $i \in \{1,2\}$ und Permutation $\phi$ auf $V_1$.
Übermittle $G_3 = \phi(G_i)$ an den Prover.

Prover antwortet mit einer Zahl $j \in \{1,2\}$. Falls $i=j$, dann akzeptieren, sonst verwerfen.

Wenn $G_1 \not\cong G_2$, so kann der intendierte Prover so antworten, dass
$V$ mit Sicherheit akzepiert.
\\
Wenn $G_1 \cong G_2$: Falls $G_3$ übermittelt wird und $P$ richtig rät, so ist
das in 50\% der Fälle die falsche Antwort, da ja
$G_3 = \phi(G_1) = \phi(\psi^{-1}(G_2))$
wenn $\phi_1: G_1 \rightarrow G_2$ Isomorphismus
\\
$\Rightarrow$ Akzeptanzwahrscheinlichkeit $\leq$ 50\%.



% TODO tex these with tikz

%\begin{beispiel}

    %G_1      1 - 2
             %| / |
             %3 - 4

    %G_2      1   2
             %| X |
             %3 - 4

       %sind isopmorph

%\end{beispiel}


%\begin{beispiel}

    %G_1      1 - 2
             %| \
             %3   4

    %G_2      1 - 2
             %| /
             %3   4

       %sind nicht isopmorph

%\end{beispiel}





\subsection{Unerfüllbarkeit von 3KNF}
\label{sec:unerfullbarkeit_von_3knf}


Ziel: $\mathit{co-NP} \subseteq \mathit{IP}$

Konstruktion eines IP-Protokolls für Unerfüllbarkeit von 3KNF.

Vorgelegt sei eine 3KNF $\phi$.
\\
Gefragt ist $\phi(\eta) = 0$ für alle Belegungen $\eta$.
\\
$\eta: Vars(\phi) \rightarrow \{0.1\}$


Lösung ist die Arithmetisierung.
Man ordnet jeder KNF $\phi$ ein Polynom $[\phi]$ in denselben Variablen zu.
\begin{itemize}
    \item $[ x ] = x$
    \item $[ \phi \land \psi ] =  [\phi] * [\psi]$
    \item $[ \phi \lor \psi ] =  1 - (1 - [\phi]) * (1 - [\psi])$
    \item $[ \neg \phi ] =  1 - [\phi]$
\end{itemize}

Falls $\eta: Vars \rightarrow \{0,1\}$ \\
$\phi [\eta] = [\phi](\eta)$ \\
Durch Induktion über Aufbau von $\phi$.

$\phi = (x \lor \neg y) \land z$ \\
$[\phi] = (1 - (1-x) (1-(1-y))) * z$


Wenn $\phi$ eine 3KNF mit $n$ Variablen und $m$ Klauseln ist, dann ist $[\phi]$
ein Polynom in $n$ Variablen (denselben) und vom Grad $3m$.





Wir erweitern die Syntax von Polynomen um den Operator
$SUM_x$ wobei $SUM_x(p) = p(x=0) + p(x=1)$.

Jetzt ist $\phi$ unterfüllbar (mit Variablen $x_1 \dots x_2$) genau dann, wenn
\\
$SUM_{x_1}(SUM_{x_2}(SUM_{x_3}(\dots  SUM_{x_n}([\phi]) )))$.

Wir entwerfen ein IP-Protokoll für das Problem $p(a_1 \dots a_n) = b$?,
wobei $p$ verallgemeinertes Polynom ist.

Alle Zwischenergebnisse bei der Auswertung eines erweiterten Polynoms
sind durch $2^m$ beschränkt (und größer oder gleich 0).\\
Wir können daher alle Berechnungen in $\Z / p\Z (\Z_p), (\mod p)$ durchführen
(statt in $\R$), sofern $p$ eine Primzahl $\geq 2^m$ ist.

Ganz zu Beginn einigen sich $P$ und $V$ also auf solch eine Primzahl $p$.
Anschließend finden alle Berechnungen $\mod p$ statt.
Das Protokoll läuft über so viele Runden, wie das verallgemeinerte Polynom $p$ groß ist.


Falls $p = x_i$: Verifizierer prüft, ob $a_i = b$. Wenn ja, akzeptiere sonst verwerfen.
\\
Falls $p = p_1 * p_2$: Verifizierer fragt Prover nach Zwischenergebnissen
$b_1, b_2$ und prüft, ob $b_1 * b_2 = b$ und in folgenden Runden wird dann
bestätigt, dass $p_1(a) = b_1, p_2(a) = b_2$.



Fülle $p = p_1 + p_2$, $p = P_1 - P_2$ \\
$p = 1$ analog.

Fall $p = SUM_x p_1(x_1 \dots x_n, x)$
\\
Gefragt ist $p_1(a_1 \dots a_n, 0) + p_1(a_1 \dots a_n, 1) = b$.
\\
Verifizierer bittet Prover um die Koeffizienten (in $\Z_p$) es Polynoms
(vollständig ausmultipliziert.)
\\
$q(x) = p_1(a_1 \dots a_n, x)$ Polynom in 1 Variablen vom Grad $3m$.
\\
Verifizierer prüft, ob $q(0) + q(1) = b$.
\\
Verifizierer wählt zufällig $a \in \Z/p\Z$.
\\
In weiteren Runden wird bestätigt, dass
$p_1(a_1 \dots a_n, a) = q(a)$

Das funktioniert intuitiv deshalb weil bei Polynomen eine zufällige Einsetzung
fast so gut wie eine symbolische Variablenbelegung ist.










\datum{28.01.16}


\textbf{IP-Protokoll:}

Zu Beginn jeder Runde liegt ein verallgemeinertes Polynom $p$ in Variablen $x_1
\dots x_n$ vor und Werte $a_1 \dots a_k \in \Z_p$, wobei $p$ eine Primzahl $\geq
2^n$ ist.

$b \in \Z_p$
\\
Es gibt dann zu bestätigen, dass $p(a_1 \dots a_k) = b$.
\\
Gestartet wird mit
$SUM_{x_1} \dots SUM_{x_n}[q] \stackrel{?}{=} 0$ für $k = 0$.
\\
Ist $p(a_1 \dots a_k) \equiv SUM_x p_1(a_1 \dots a_k, x)$
\\
* Wählt $V a \in \Z_p$ zufällig und lässt sich das Polynom
$q_(x) = p_1(a_1 \dots a_k, x)$
von P geben.
\\
$V$ prüft, ob $q(0) + q(1) = b$ und in der nächsten Runde wird bestätigt, dass
$p_1(a_1 \dots a_k, a) = q(a)$
\\
Falls $\phi$ unerfüllbar ist, dann kann P alle Fragen wie gewünscht beantworten und
$V$ wird zu 100\% überzeugt.
Falls $phi$ erfüllbar ist, ist die einzige Möglichkeit, dass P den Verifier
fälschlicherweise überzeugt, dass ein oder mehrere Male ander Stelle * zufällig
eine der $\leq 3m$ Nullstellen von $p_1(a, x) - q(x)$ wobei $p_1(a, x) \neq
q(x)$, erwischt wurde.

Die Wahrscheinlichkeit, dass das nicht passiert, ist $\geq (1 -
\frac{3m}{p})^2$. (Der Bruch bezeichnet die Größe des Körpers $\Z_p$)
\\
$p \geq c^n \leadsto \text{ Wahrscheinlichkeit } \geq 75\%$ wie gewünscht.

%TODO begründung dazu kommt am montag





\textbf{Verallgemeinerung:}


$\mathit{IP} \supseteq \mathit{PSPACE}$

$\mathit{QBF} \in \mathit{IP}$

Verallgemeinerte Polynome werden um zusätzliche Operatoren erweitert:
\begin{itemize}
    \item $\mathit{EX}_x p(x) = 1 - (1 - p(0))(1 - p(1))$
    \item $\mathit{ALL}_x p(x) = p(0) * p(1)$
    \item $\mathit{RED}_x p(x) = p(0) + x * (p(1) - p(0))$
\end{itemize}
$\mathit{RED}_x p(x)$ stimmt mit $p(x)$ auf $[0,1)$ überein.
$\mathit{RED}_x p(x)$ hat in $x$ aber Grad 1.

Einer QBF-Formel $\phi$ kann man mit diesen Operatoren ein verallgemeinertes
Polynom $[\phi]$ vom Grad $n$ (Anzahl der Variablen in $\phi$) zuordnen, sodass
$[\phi] = 1 \Leftrightarrow \phi \text{ wahr (keine freien Variablen)}$.




\begin{satz}

Zusammengefasst gilt:
$\mathit{IP} = \mathit{PSPACE}$

Es gibt aber Orakel $A$, sodass
$\mathit{IP}^A \neq \mathit{PSPACE}^A$

\end{satz}








\subsection{Probabilistically Checkable Proof}
\label{sec:probabilistically_checkable_proof}


\begin{definition}
    Bei \definiere{Probabilistically Checkable Proof} (PCP) gibt es einen
    Verifier wie bei IP.
    Statt eines interaktiven Provers gibt es jetzt einen ``Beweis'' $\Pi \in
    \{0,1\}^N$.
    $V$ stellt Fragen an $\Pi$ in Form eines Bitstrings $q \in \{0,1\}^{\log
    N}$.
    Diese werden mit dem $q.$ Bit von $\Pi$ beantwortet.

    \begin{align*}
        L \in \mathit{PHP} &\\
        \Leftrightarrow \exists V . & \\
                                    & x \in L \Rightarrow \exists \Pi . (\Pi, V) \text{ akzepiert mit
    Wahrscheinlichkeit } \geq 75\% \\
                                    & x \notin L \Rightarrow \forall \Pi . (\Pi, V) \text{ akzepiert mit
    Wahrscheinlichkeit } \leq 25\%
    \end{align*}

\end{definition}

Falls Rechenzeit von $V$ $\leq p(n)$, wobei $n = |x|$ und $p$ Polynom,
dann machen nur Beweise $\Pi$ mit $|\Pi| \leq 2^{p(n)}$ Sinn.

Offensichtlich ist $\mathit{IP} \subseteq \mathit{PCP}$.
Die Umkehrung ist fraglich, denn $\mathit{PCP} = \ComplexityClassNEXP$. Warum
ist nicht offensichtlich $\mathit{PCP} \subseteq \mathit{IP}$.

Einschränkungen:
\begin{itemize}
    \item $\mathit{PCP}(\psi(n), q(n)) = \{ L | \exists \mathit{PCP}
            \text{ Verifier } V $, der bei Eingabe $x$ nur $\bigO(\psi(|x|))$
            Zufallsbits verwendet und $\bigO(q(|x|))$ Anfragen stellt.
            $\}$
\end{itemize}



\begin{satz}
    \textbf{PCP-Theorem}

    $\ComplexityClassNP = \mathit{PCP}(\log(n), 1)$

    $c * \log(n)$ Zufallsbits
    \\
    $d$ Fragen
    \\
    Insgesamt $2^{c * \log(n)} * d = n^c * d = poly(n)$

    Der Beweis hat polynomielle Größe, kann also geraten werden.
    \\
    $\Rightarrow P(P(\log(n), 1)) \subseteq \ComplexityClassNP$

\end{satz}


Im Buch ist ein Beweis von
$\ComplexityClassNP \subseteq \mathit{PCP}(poly(n), 1)$








\textbf{ Problem MAX-CLIQUE-APPROX }

Gegeben ein Graph $G = (V, E)$.

Man berechne eine Clique in $G$, die mindestens halb so groß ist, wie die
maximale Clique.
Das Problem nennt man MAX-CLIQUE-APPROX (MCA).


\begin{satz}
    Falls $\ComplexityClassP \neq \ComplexityClassNP$, dann gibt es keinen
    polynomiellen Algorthmus für das Problem MAX-CLIQUE-APPROX.
\end{satz}

\begin{beweis}
    Wir geben uns einen $\mathit{PCP}(\log(n), 1)$ Verifier $V$ für SAT vor.
    Außerdem einen hypothetischen Polynomialzeit-Algorithmus für MCA.

    Daraus bauen wir ein Polynomialzeit-Entscheidungsverfahren für SAT.

    Sei 3KNF $x$ vorgelegt.
    Es werden $d = \bigO(\log(|x|))$ Zufallsbits verwendet und $t = \bigO(1)$
    Fragen gestellt.

    Transcript:
    $(r, q_1, a_1, q_2, a_2, \dots q_t, a_t)$,
    wobei $r$ die Zufallsbits und $q_i$ die Fragen sind.
    Dieses Transcript beschreibt einen möglichen Lauf der Berechnung von $V$.

    Nachdem $V$ eine Turingmaschine ist, hängen die $q_i$ nur ab von $r$ und
    $a_1 \dots a_t$.
    Gültige Transcripts sind nur solche, die tatsächlich durch $V$ in
    Erscheinung treten.
    Die Anzahl der Transcripts ist $2^d * 2^t = poly(n)$.

    Bilde Graphen $G = (V, E)$, wobei $V$ die Menge der Transcripts und $E$ die
    Konsistenz der Transcripts ist.

    $(r, q_1, a_1, q_2, a_2, \dots q_t, a_t)$ und
    $(r', q_1', a_1', q_2', a_2', \dots q_t', a_t')$
    sind konsistent, wenn $q_i = q_i' \Rightarrow a_i = a_i'$.

    Jeder funktionierende Beweis $\Pi$ liefert eine Clique der Größe $\geq 0.75
    * 2^d$, wobei $d$ so in etwa $\bigO(\log(i))$ ist oder so.
    \\
    Umgekehrt gibt es keinen Beweis mit Akzeptanzwahrscheinlichkeit $\geq 25\%$,
    dann gibt es keine Clique $\geq 0.25 * 2^d$.

    Mit dem hypothetischen Algorthmus für MCA berechnet man eine Clique, die
    mindestens halb so groß wie die größte Clique ist.
    Falls $> 0.5 * 0.75 * 2^d = 0.375 * 2^d$, dann akzeptieren, sonst verwerfen.

\end{beweis}






\textbf{Schaltkreiskomplexität}

Schaltkreise aus UND-, ODER-Gattern. NICHT-Gatter stellen

XOR mit NAND gebaut:
\url{https://www.circuitlab.com/circuit/5m6zgn/xor-gate-with-nand-gates/}


UND-Gatter gefolgt von UND können verschmolzen werden. ODER-Gatter auch. (Also
mehr Eingänge statt hintereinander stecken.)

Daher können wir Schaltkreise als geschichtet annehmen, wobei jede Schicht
Gatter derselben Art enthält und UND, ODER abwechseln.

Die Größe eines Schaltkreises ist die Anzahl der Gatter.
\\
Die Tiefe eines Schaltkreises ist die Anzahl der Schichten
(Länge des längsten Pfades von Blatt zu Wurzel).


DNF:\\
Tiefe 3 genügt, wenn man exponentielle Größe in Kauf nimmt.


Satz: First-Save-Sipser oder so:\\
Paritätsfunktion ($\Sigma \mod 2$) kann nicht mit polynomieller Größe und
konstanter Tiefe berechnet werden.
\\
(Tiefe $\log n$ leicht: Binärbaum aus XOR).













\datum{01.02.16}



\textbf{Random Fact für Übung}

$(1-a)^n \geq 1 - an$ für $a \geq 0$



\textbf{zu Schaltkreisen}

Man kann Schaltkreise o.B.d.A nur aus $\land$ und $\lor$ bauen.

\begin{itemize}
    \item
        $AC_0$: konstante Tiefe, polynomielle Größe (polynomiell in $n$, der Anzahl der Variablen).
    \item
        $NC_0$: Wie $AC_0$ aber Fan-In $= 2$.
    \item
        $AC_k$: Tiefe in $\bigO((\log n)^k)$, polynomielle Größe
    \item
        $NC_k$: Wie $AC_k$ aber Fan-In $= 2$.
\end{itemize}

$AC_k \subseteq \ComplexityClassP$, sogar wenn polynomielle Tiefe erlaubt wäre.

$AC_k \subseteq NC_{k+1}$: Ersetze UND-Kette (Länge $n^d$) durch UND-Baum (Tiefe $\bigO(\log n))$.

$NC_k \subseteq AC_k$



\begin{satz}
    \textbf{Furst-Saxe-Sipser}

    $\text{PARITY} \notin AC_0$

    $\text{PARITY}(x_1 \dots x_n) =
    \begin{cases}
        1   & , \text{falls } x_1 + \dots x_n \mod 2 = 1 \\
        0   & , sonst
    \end{cases}
    $

    Offensichtlich $\text{PARITY} \in NC_1$:
    Binärbaum aus XOR-Gattern der Tiefe $\log n$.
    XOR durch 4 NAND-Gatter ersetzen usw. liefert Behauptung.


\end{satz}

\begin{beweis}
    \textbf{von Razborow-Smolerity oder sowas keine Ahnung}

    Ausgangspunkt:
    Darstellung von Schaltkreisen durch Polynome:
    \begin{itemize}
        \item $[x] = x$
        \item $[\neg x] = 1 - x$
        \item $[s_1 \land \dots \land s_d] = [s_1] * [s_2] * \dots * [s_d]$
        \item $[s_1 \lor \dots \lor s_d] = 1 - (1 - [s_1]) * \dots * (1 - [s_d])$
        \item $s(\eta) = [s](\eta) \forall \eta \in \{0,1\}^n$
    \end{itemize}
    Jede Funktion aus $AC_0$ kann also durch ein Polynom dargestellt werden mit
    Grad $poly(n)$.
    \\
    $poly(n)^t = poly(n)$ falls $t$ konstant.
    \\
    Dies gilt also auch für $\text{PARITY}$ falls $\text{PARITY} \in AC_0$, aber
    daraus lässt sich noch kein Widerspruch ableiten.
    \\
    Es gilt aber auch: $\forall s \in AC_0: \exists \text{ Polynom } p_s: Grad
    \frac{1}{2}\sqrt{n} \land s(\eta) = p_s(\eta) \text{ für } 90\% \text{ der }
    \eta \in \{0,1\}^n$.

    Gute Approximation der $v$-Funktion auf $n$ Variablen und kleinem Grad:
    $S_0 = \{1, \dots n\}$
    \\
    $S_0 \supseteq S_1 \supseteq  S_2 \supseteq  S_3 \dots \supseteq S_l  $
    \\
    Teilmengen $S_i$ werden durch Zufallsprozess erzeugt.
    $P(i \in S_{k+1} \ |\ i \in S_k) = \frac{1}{2}$
    \\
    $l = \log(n) + 2$
    \begin{itemize}
        \item $S_0 = \{1, 2, 3, 4, 5\}$
        \item $S_1 = \{1, 2, 5\}$
        \item $S_2 = \{2, 5\}$
        \item $S_3 = \{2\}$
        \item $S_x = \emptyset$
    \end{itemize}
    Mit diesen $S_0, S_1 \dots S_{\log(n)+2}$ (durch Zufallsprozess gebildet.)
    wird jetzt folgendes Polynom konstruiert:
    $q(x_1 \dots x_n) = 1 - (1 - q_0(x)) (1 - q_1(x)) \dots (1 - q_l(x))$
    \\
    $q_i(x_1 \dots x_n) = \Sigma_{j \in S_i} = x_j$
    z.B. $q(x_1 \dots x_5) = 1 - (1 - (x_1 + x_2 + x_3 + x_4 + x_5)) (1 - (x_1 +
    x_2 + x_5)) (1 - (x_2 + x_5)) (1 - x_2)$
    \\
    Falls alle $x_i$˘Null ist, so ist $q(x) = 0$.

    Sei jetzt $T = \{i | x_i \neq 0 \}$ z.B. $T = \{2,5\}$.
    Im Beispiel ist dann tatsächlich $q_(x) = 1$.
    Für $T' = \{3,4\}$ allerdings ergibt sich $q(x_1 \dots x_5) = 2$.
    \\
    Für $x_i \in \{0,1\}$ gilt $q(x_1 \dots x_n) = x_1 \lor \dots \lor x_n$
    falls entweder $T = \emptyset$ oder ein $i \leq l$ existiert mit $| S_i \cap
    T | = 1$,
    dann ist $q_i(x) = 1$.

    Die Wahrscheinlichkeit, dass das passiert, ist $\geq \frac{1}{2}$, wenn
    mindestens ein $x_i = 1$.
    \\
    $P(|S_l \cap T| > 1) \leq P(|S_l \cap T| \geq 1) \leq P(|S_l| \geq 1) \leq n
    * 2^l = n * 2^{-\log(x) + 2} \leq \frac{1}{4}$
    \\
    Mit Wahrscheinlichkeit $\geq \frac{3}{4}: |S_l \cap T| \leq 1$
    \\
    Entweder ist $S_0 \cap T| = |T| = 1$ (sowieso günstig) oder $S_0 \cap T| >
    1$ und $|S_i \cap T| \leq 1$.
    \\
    $P(|S_i \cap T| = 1) = P('=1' | '\leq 1') = P('=1') / P('\leq 1') = t *
    2^{-t} / (2^{-t} + t * 2^{-t}) = t / (1 + t) \geq 2/3$
    \\
    $t = |S_{i-1} \cap T|$
    \\
    Insgesamt $\frac{3}{4} * \frac{2}{3} = \frac{1}{2}$

    $q$ zufällig konstruiert.

    Gegeben $x$: Falls $x = 0$, dann ist $q(x) = 0$, falls $x = \neq 0$, dann
    ist $P(q(x) = 1) \geq \frac{1}{2}$.

    Konstruiere jetzt $s \in \N$ viele solche Polynome $q^{(1)} \dots q^{(s)}$
    und bilde
    $q'(x) = 1 - (1 - q^{(1)}) (1 - q^{(2)}) \dots (1 - q^{(s)})$
    \\
    $Grad(q') = s * Grad(q) = s * (\log(n) + 2)$
    Fehlerwahrscheinlichkeit für $q'$ ist $2^{-s} \leq \epsilon$ falls $s =
    \log(\frac{1}{\epsilon})$.

    Gegeben Schaltkreis mit konstanter Tiefe $t$ und Größe $S = poly(n)$.
    Ersetze jedes Gatter durch so ein Zufallspolynom mit
    Fehlerwahrscheinlichkeit $\leq \epsilon$ also Grad $\log(\frac{1}{\epsilon})
    * (\log(n) + 2)$.
    Daraus resultiert ein Polynom vom Grad $(\log(\frac{1}{\epsilon}) * \log(n)
    + 2)^t $ und Fehlerwahrscheinlichkeit $1 - (1 - \epsilon)^S \leq S:\in$
    %% TODO DAFUQ

    Gewünschte Fehlerwahrscheinlichkeit ist $\leq 10\%$, $\epsilon \leq
    \frac{0.1}{S}$
    Also Grad es Polynoms $(\log(\frac{S}{0.1}) * (\log(n) + 2))^t$
    \\
    Wenn $s = poly(n)$ und $t = \bigO(1)$, dann $\log(n)^k$ für große $n$ und
    festes $k$.
    \\
    $\log(n^d) = d * \log(n)$
    \\
    Es gibt also auch mindestens eine Wahl der Zufallsentscheidungen, die für
    mindestens 90\% aller Eingaben günstig ist. Test-Verdrahten liefert festes
    Polynom $q$ vom Grad $\log(n) \bigO(1)$, welches auf mindestens 90\% der
    Eingaben richtig ist.

    Lalalalala

    Wenn jetzt $\text{PARITY } \in AC_0$, gäbe es $q'(x_1 \dots x_n)$ mit Grad
    $\frac{\sqrt{n}}{2}$, sodass für große $n$ und $S' = \{x_1 \dots x_n) |
        q'(x_1 \dots x_n) = \text{ PARITY }(x_1 \dots x_n) \}$ erfüllt $|S'|
        \geq 0.9 * 2^n$.


    \textbf{Fourier-Repräsentation}

    \begin{itemize}
        \item true ist -1
        \item false ist 1
    \end{itemize}
    Warum auch nicht.

    Wir rechnen jetzt $q'$ auf die Fourier-Repräsentation um mit $q$
    \\
    $q(x_1 \dots x_n) = 1 - 2 * q'(\frac{1-x_1}{2}, \frac{1-x_2}{2}, \dots
    \frac{1-x_n}{2})$ und der Grad ist $q = \frac{\sqrt{n}}{2}$.
    \\
    $S = \{(x_1 \dots x_n) | x \in \{-1,1\}^n . q(x) = x_1 * \dots * x_n \}$
    erfüllt $|S| \geq 0.9 * 2^n$.
    \\
    O.B.d.A. kann $q$ als multilinear vorausgesetzt werden.
    Jede Variablenbelegung nur in Potenz 1 (Also ein $x_i$ mehrfach ist böse),
    denn auf $\{-1, 1\}$ ist $x^2 = 1$.
    \\
    Zu jedem multilinearem Polynom $p(x_1 \dots x_n)$ kann mithilfe von $q$ ein
    Polynom $p'$ konstruiert werden, welches mit $p$ auf ganz $S$ übereinstimmt,
    aber Grad $\frac{n}{2} + \frac{\sqrt{n}}{2}$ hat:
    Ersetze Monom $\Pi_{i \in M} x_i: M \subseteq \{1, \dots n\}$ mit $|M| \geq
    \frac{n}{2}$ durch $q(x) * \Pi_{i \notin M} x_i$.


\end{beweis}



Sei $U$ der Vektorraum der multilinearen Polynome in $x_1 \dots x_n$ vom Grad
$\leq \frac{n + \sqrt{n}}{2}$
\\
$dim(U) = \Sigma_{i=0}^{(n+\sqrt{n})/2} (n über i) \leq 0.88 * 2^n$ durch die
Stirling-Approximation. Und das gibt dann, wie man ja offensichtlich sieht - wer
tut das nicht - einen Widerspruch zu dem Kram davor.

Dann definieren wir noch für $s \in S$ irgendsoein $k_x(x)$, das man per DNF und
so als multilineares Polynom darstellen kann und wegen der Reduktion gibt es
dann auch zu jedem $s \in S$ ein multilineares Polynom aus $U$ mit $k_S(x) =
k'_s(x)$ für alle und die $k$ sind alle linear unabhängig in $U$ und ihre Zahl
ist $\geq 0.9 * 2^n$ und das können wir jetzt wirklich selber ausrechnen, er hat
es im Kopf und schreibt es jetzt nicht mehr auf. Joa.











% chapter probabilistische_algorithmen (end)

%%% Local Variables:
%%% mode: latex
%%% TeX-master: "0-main"
%%% ispell-local-dictionary: "de-neu"
%%% eval: (highlight-parentheses-mode)
%%% End:
