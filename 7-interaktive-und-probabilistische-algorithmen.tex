%!TEX root = 0-main.tex

% Author: Philipp Moers <soziflip@gmail.com>



\datum{07.01.16}


\chapter{Probabilistische Algorithmen} % (fold)
\label{cha:probabilistische_algorithmen}



\section{Miller-Rabin Primzahltest Einführung}

Gegeben: $n \in \N$ binär kodiert, $\text{Länge}(n) = \log(n)$

Gefragt: Ist $n$ prim?

Man müsste testen bis $\sqrt{n}$
Anzahl der Zahlen $< \sqrt{n}$: $\sqrt{2^{\log(n)}} = 2^{\frac{1}{2} \log(n)}$, d.\,h. exponentiell in $\text{Länge}(n)$.
Das ist nicht effizient, insbesondere $\notin \ComplexityClassP$.
% todo wo ist das P


\begin{lemma}

    Sei $G$ abelsche Gruppe (endlich) und $U$ eine Untergruppe von $G$.

    Dann ist $|U|$ ein Teiler von $|G|$.

\end{lemma}

\begin{korollar}

    Falls $ggt(x, n)$ = 1, d.\,h. $x \in \Phi(n)$, dann $x^{\phi(n)} = 1 \mod n$,
    wobei $\phi(n) = |\Phi(n)|$ die Anzahl der zu $n$ teilerfremden Zahlen kleiner $n$ ist.

\end{korollar}

\begin{korollar}

    Falls $n$ prim ist, so gilt $x^{n-1} = 1 \mod n$

\end{korollar}


\begin{lemma}

    Falls $x \notin \Phi(n)$, so ist $x^{n-1} \neq 1 \mod n$.

\end{lemma}
\begin{beweis}

    Sei $b = ggt(x, n) > 1$, $v = x^{n-2} \mod n$.

    $x^{n-1} = v * x + u * n $ für ein $u \in \Z$

    $b | RHS \Rightarrow x^{n-1} \neq 1$

\end{beweis}



\section{Fermat-Primzahltest}

Gegeben: $n \in \N$

Wähle zufällig $x \in \{1 \dots n-1\}$.
\\
Bilde $a := x^{n-1} \mod n$ mittels ``Repeated Squaring''.
\\
Falls $a = 1$, so antworte ``ist prim''.
\\
Falls $a \neq 1$, so antworte ``ist zusammengesetzt''.

Ist $n$ tatsächlich prim, so wird mit Sicherheit geantwortet ``ist prim'' (siehe Korollar).
\\
Ist $n$ zusammengesetzt und $x \notin \Phi(n)$, dann wird korrekt geantwortet ``ist zusammengesetzt''.
\\
Wenn $x \in \Phi(n)$, so kann trotzdem $a = 1$ sein, also korrekt geantwortet werden.

Falls $U \neq \Phi(n)$, dann ist $|U|$ ein echter Teiler von $phi(n)$, also $\leq \frac{1}{2} n$.
D.\,h. für mehr als 50\% der $x$ ist dann $a \neq 1$.

Falls aber für alle $x \in \Phi(n)$ gilt $x^{n-1} = 1 \mod n$ und doch $n$ zusammengesetzt ist, dann funktioniert der Fermat-Test nicht gut. Solche $n$ heißen Carmichaelzahl, falls $n$ nicht prim, aber $x^{n-1} = 1 \mod n \forall x \in \Phi(n)$. Solche Carmichaelzahlen sind sehr selten.



\begin{satz}

    Ist $n$ Carmichaelzahl, so ist der Anteil der $x \in \{1 \dots n-1\}$, sodass $\exists i: ggt(x^{\frac{n-1}{2^i}} - 1, n) > 1$ größer oder gleich 75\%.

\end{satz}



\section{Miller-Rabin Primzahltest}

Gegeben $n \in \N$

Wähle zufällig $x \in \{1 \dots n-1\}$.
\\
Bilde $a = x^{n-1} \mod n$.
\\
Falls $a \neq 1$, so ist $n$ sicher zusammengesetzt.
\\
Falls $a = 1$, berechne $ggt(x^{\frac{n-1}{2^i}} - 1, n)$ für $i = 1 \dots \log(n)$. Falls einmal $> 1$, so ist $n$ sicher zusammengesetzt. Sonst antworte ``ist prim''.

Fehlerwahrscheinlichkeit $\leq 50\%$.





\section{Testen von polynomiellen Identitäten}

% TODO vbl?
Gegeben: zwei Polynome $p(x), q(x)$ mit Koeffizienten aus $\Z$ und $n$ Vbl $x = x_1 \dots x_n$.

Gefragt: Ist $p(x) = q(x)$?


\begin{beispiel}

    $p = (a^2 + b^2 + c^2 + d^2)(A^2 + B^2 + C^2 + D^2)$

    $q = (aA + bB + cC + dD)^2 + \\
         (aB + bA + cD + dC)^2 + \\
         (aC + bD + cA + dB)^2 + \\
         (aD + bC + cB + dA)^2 \\
    $

    Hier ist in der Tat $p = q$.

\end{beispiel}

Für diese Aufgabe ist kein Polynomialzeitverfahren bekannt.
Ausmultiplizieren hat im Allgemeinen exponentielle Laufzeit, falls $*$ und $+$ geschachtelt auftreten.



\textbf{Probabilistisches Verfahren:}

Wähle zufällig $x \in \{-nd, \dots 0, 1, 2, \dots , nd\}^n$,
wobei $n = \text{ Anzahl Vbl } $ und $  d \geq Grad(p), Grad(q)$.

Wir zeigen jetzt, dass die Fehlerwahrscheinlichkeit kleiner oder gleich 50\% ist.

\begin{lemma}

    \definiere{Lemma von Schwarz-Zippe}

    Sei $p$ ein Polynom vom Grad $d$ in $n$ Variablen und $p \neq q$.

    Dann hat $p$ im Bereich $\{-N, \dots N\}^n$ höchstens $dn(2N+1)^{n-1}$ Nullstellen.

\end{lemma}

Falls $N = nd$, dann ist $\frac{(2N+1)^{n-1} * dn}{(2N+1)^n}  =  \frac{dn}{2N+1} = \frac{dn}{2dn+1} \leq 50\%$.








\datum{11.01.16}


\begin{definition}

    Ein \definiere{Perfektes Matching} zu Graph $G= (V, E)$ ist eine
    Kantenauswahl $T \subseteq E$, sodass
    $u, v \in T, u \neq v \rightarrow u \cap v = \emptyset$.
    (Die ausgewählten Kanten haben keine Punkte gemeinsam.)

    Ein Graph ist perfekt, wenn $\bigcup T = V$.
    (Alle Knoten werden gematcht.)

\end{definition}


\begin{satz}
    Sei $G=(V,E)$ ein Graph und $A$ die zugehörige Adjazenzmatrix.
    $G$ hat ein perfektes Matching genau dann, wenn $det(A) \neq 0$.
\end{satz}



\begin{definition}
    Eine \definiere{Probabilistische Turingmaschine} ist wie eine nichtdeterministische
    Turingmaschine aufgebaut. Zu jedem Paar $q, a$ gibt es exakt zwei Quintupel
    (ggf.\ künstlich aufblähen).
\end{definition}

Deutung:
Passendes Quintupel wird jeweils mit Wahrscheinlichkeit 50\% ausgewählt.
Die Wahrscheinlichkeit, dass $T$ die Eingabe $x$ akzeptiert, ist
$\frac{\text{Anzahl akzeptierender Berechnungen}}{\text{Anzahl aller Berechnungen}}$.

Polynomialzeitbeschränkt, falls Laufzeit $\leq p(|x|)$ unabhängig von den Zufallsentscheidungen.


\begin{definition}
    Eine Sprache $X \subseteq \Sigma^\ast$ ist in der Komplexitätsklasse $\ComplexityClassPP$,
    wenn eine Probabilistische Turingmaschine $T$ existiert, die
    \begin{itemize}
        \item polynomiell zeitbeschränkt ist,
        \item $x \in X \Leftrightarrow \text{Akzeptanzwahrscheinlichkeit} \geq 50\%$.
    \end{itemize}
\end{definition}

\begin{satz}
    $\ComplexityClassP \subseteq PSPACE$
\end{satz}
\begin{beweis}
    Alle Berechnungen durchprobieren (Platz jeweils wiederverwenden) und Anzahl der
    akzeptierenden Berechnungen mitzählen.
\end{beweis}

\begin{satz}
    $\ComplexityClassNP \subseteq PSPACE$
\end{satz}
\begin{beweis}
    Sei eine nichtdeterministische Turingmaschine $T$ für $X$ gegeben
    (O.B.d.A. im 2. Nachfolgequintupelformat).
    Zunächst mit 50\% Wahrscheinlichkeit einfach so akzeptieren,
    anschließend $T$ als Probabilistische Turingmaschine fahren.
    \\
    $x \in X$: Wahrscheinlichkeit der Akzeptanz $\geq 50\%$
    \\
    $x \notin X$: Wahrscheinlichkeit der Akzeptanz $= 50\%$
\end{beweis}

\begin{satz}
    (Beispiel Spielman et al.)

    $\ComplexityClassPP$ ist unter $\cap, \cup$ abgeschlossen.
\end{satz}
Der Beweis dazu ist kompliziert.



\begin{definition}
    Komplexitätsklassen $\ComplexityClassR, \ComplexityClassRP$

    $X \in \ComplexityClassR$ genau dann, wenn eine polynomiell zeitbeschränkte Probabilistische Turingmaschine $T$ existiert mit
    \begin{itemize}
        \item $x \in X \Rightarrow$ Wahrscheinlichkeit der Akzeptanz $\geq 50\%$
        \item $x \notin X \Rightarrow$ Wahrscheinlichkeit der Akzeptanz $= 0$
    \end{itemize}
\end{definition}

\begin{beispiel}
    $PRIM \in \ComplexityClassR$
\end{beispiel}


\begin{definition}
    Komplexitätsklasse $\ComplexityClassBPP$

    $X \in \ComplexityClassBPP$ genau dann, wenn eine polynomiell zeitbeschränkte Probabilistische Turingmaschine $T$ existiert mit
    \begin{itemize}
        \item $x \in X $: Wahrscheinlichkeit der Akzeptanz $\geq 75\%$
        \item $x \notin X $: Wahrscheinlichkeit der Akzeptanz $\leq 75\%$
    \end{itemize}
\end{definition}

\begin{satz}
    Sei $A \in \ComplexityClassBPP$ und $q(n)$ ein Polynom.
    \\
    Es existiert eine Probabilistische Turingmaschine $T$ für $A$
    mit Fehlerwahrscheinlichkeit $\leq e^{-q(|x|)}$ für Eingabe $x$.
\end{satz}
\begin{beweis}
    Beweisidee: $A$ auf Eingabe $x$ wiederholt, d.\,h. $t = t(|x|)$ mal ablaufen lassen.
    Am Ende diejenige Antwort, die häufiger gegeben wurde, nach außen reichen.

    $x \in A$ Anzahl der Akzeptanzen binomialverteilt mit Parameter $p \geq 75\%$.
    Sei $S$ diese Anzahl (Zufallsvariable).
    Wir möchten $Pr(S \leq \frac{t}{2}) \leq e^{-q(|x|)}$.

    Es gilt allgemein die Chernoff-Schranke.
    (\textit{siehe Details im Buch})

\end{beweis}







% chapter probabilistische_algorithmen (end)
