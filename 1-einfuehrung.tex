%!TEX root = 0-main.tex

% Author: Philipp Moers <soziflip@gmail.com> 



\chapter{Einführung} % (fold)
\label{cha:einfuhrung}



\section{Motivation}

Theoretische Informatik, Berechenbarkeit und insbesondere Komplexitätstheorie ist \emph{der} Informatiker-Shit schlechthin. Let's do it!



\section{Literatur}

Die Vorlesung basiert hauptsächlich auf folgendem Buch:
\begin{itemize}
    \item Bovet, Crescenzi. Introduction to the Theory of Complexity. Prentice Hall. New York. 1994.
\end{itemize}

Weiterhin ist folgende Literatur gegeben:
\begin{itemize}
    \item C. Papadimitriou. Computational Complexity. Addison-Wesley. Reading. 1995.
    \item I. Wegener. Komplexitätstheorie: Grenzen der Effizienz von Algorithmen. Springer. 2003.
    \item S. Arora und B. Barak. Complexity Theory: A Modern Approach. 
\end{itemize}

Zur Motivation:
\begin{itemize}
    \item Heribert Vollmer. Was leistet die Komplexitätstheorie für die Praxis? Informatik Spektrum 22 Heft 5, 1999.
    \item Stephen Cook: The Importance of the P versus NP Question. Journal of the ACM (Vol. 50 No. 1)
\end{itemize}







% chapter einfuhrung (end)
