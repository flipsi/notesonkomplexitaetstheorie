% Author: Philipp Moers <soziflip@gmail.com> 

\documentclass[12pt, oneside, a4paper, numbers=noenddot, abstracton, parskip=full]{scrreprt}

% Author: Philipp Moers <soziflip+latex@gmail.com>


\usepackage[utf8]{inputenc}
\usepackage[T1]{fontenc}

\usepackage[ngerman, english]{babel}
% document: \selectlanguage{ngerman}


\usepackage{hyphenat}

\usepackage{xifthen}

% Häkchen mit \checked
\usepackage{wasysym} 

% URLs
% \usepackage[hyphens]{url}
% hyperref loads url internally!
\PassOptionsToPackage{hyphens}{url}
\usepackage{hyperref}

% Uhrzeit mit \currenttime
\usepackage{datetime}

\newcommand{\todo}[1]   {\textcolor{red}{TODO: #1}} 


\usepackage{moreverb}



%%%%%%%%%%%%%%% Font %%%%%%%%%%%%%%%


% \renewcommand*{\familydefault}{\rmdefault}
\renewcommand*{\familydefault}{\sfdefault}
% \renewcommand*{\familydefault}{\ttdefault}


% Achtung, eigener Font im Mathematik Abschnitt!



%%%%%%%%%%%%%%% Unicode characters %%%%%%%%%%%%%%%

\usepackage{newunicodechar}
\usepackage{marvosym}

% \newunicodechar{∈}{\in}
\newunicodechar{∈}{$\in$}
\newunicodechar{€}{$\EURdig$}





%%%%%%%%%%%%%%% Farben %%%%%%%%%%%%%%%

\usepackage[usenames,dvipsnames,svgnames,table]{xcolor}
\definecolor{grau}{gray}{.90}
% \definecolor{orange}{RGB}{255,127,0}
\definecolor{myred}{HTML}{E78356}
\definecolor{myorange}{HTML}{FCD078}
\definecolor{mylightblue}{HTML}{A7CED1}
\definecolor{mylightgreen}{HTML}{9DE66E}
\definecolor{mylightpurple}{HTML}{FF9BF6}
\definecolor{mark}{HTML}{FFDDCC}


%%%%%%%%%%%%%%%	Tabellen %%%%%%%%%%%%%%%

\usepackage{tabularx}
\usepackage{colortbl}
\usepackage{multirow}
\newcolumntype{C}[1]{>{\centering\arraybackslash}m{#1}}


%%%%%%%%%%%%%%%	Layout etc. %%%%%%%%%%%%%%%

% \usepackage{enumerate}
% \usepackage{paralist} % compactenum

%%% IN JEDEM ÜBUNGSBLATT %%%
% \usepackage{sectsty}
% % \allsectionsfont{\large\underline} % überschriftendesign
% \sectionfont{\large\underline} % überschriftendesign
% \subsectionfont{\normalsize}


\usepackage{caption}
\captionsetup{margin=10pt,font=small,labelfont=bf}

\usepackage{courier}

\usepackage{pgf} \pgfmathsetmacro{\dotradius}{0.05}

\usepackage[a4paper]{geometry}

\usepackage{rotating}

\usepackage{setspace} \doublespacing

\renewcommand{\baselinestretch}{1.25}\normalsize
\renewcommand{\arraystretch}{1.0}


%%%%%%%%%%%%%%%	Übungsblatt Header %%%%%%%%%%%%%%%

% \newcommand{\header}[3]{\setlength{\parindent}{0pt}#1\hfill\today\\[12pt]\begin{huge}\textbf{#2}#3\end{huge}\\\rule{1\textwidth}{0.3pt}\\[9pt]}
\newcommand{\header}[3]
{\setlength{\parindent}{0pt}
\hfill\today\\
#1\\[18pt]
\begin{large}\textbf{#2}\end{large}\\[9pt]
\begin{huge}\textbf{#3}\end{huge}\\
\rule{1\textwidth}{0.3pt}\\[6pt]}



%%%%%%%%%%%%%%% Plots %%%%%%%%%%%%%%%

% \usepackage{gnuplottex}




%%%%%%%%%%%%%%%	Zeichnungen %%%%%%%%%%%%%%%

\usepackage{tikz}
\usetikzlibrary{decorations.markings,arrows} 
\usetikzlibrary{shapes,positioning,shadows,arrows,automata}
\tikzstyle{mystate} = [state, minimum size = 1.5cm, node distance = 2.5cm]
\tikzstyle{arrowlink} = [decoration={markings,mark=at position 0.1 with \arrow{triangle 60}}, postaction=decorate, color=black] 
\tikzstyle{nippel} = [circle, fill = myorange, draw, font=\sffamily\bfseries, minimum size=1.5cm]
\tikzstyle{bucket} = [rectangle, draw, font=\sffamily\bfseries, auto, thick, node distance = 1cm, minimum size=1cm]
\tikzstyle{decision} = [diamond, draw, fill=mark, text width=4.5em, text badly centered, node distance=3cm, inner sep=0pt]
\tikzstyle{block} = [rectangle, draw, fill=mark, text width=5em, text centered, rounded corners, minimum height=4em]
\tikzstyle{line} = [draw, -latex'] 
\tikzstyle{bplus}=[rectangle split, rectangle split horizontal,rectangle split ignore empty parts,draw]


% entity relationship diagram
% \usepackage{tikz-er2}
% \tikzstyle{every entity} = [top color=white, bottom color=myorange!30, draw=myorange!50!black!100, drop shadow]
% % \tikzstyle{every weak entity} = [drop shadow={shadow xshift=.7ex, shadow yshift=-.7ex}]
% \tikzstyle{every attribute} = [top color=white, bottom color=mylightblue!20, draw=mylightblue, node distance=1cm, drop shadow]
% \tikzstyle{every relationship} = [top color=white, bottom color=myred!20, draw=myred!50!black!100, drop shadow]
% % \tikzstyle{every isa} = [top color=white, bottom color=green!20, draw=green!50!black!100, drop shadow]


%%%%%%%%%%%%%%% Programmcode %%%%%%%%%%%%%%%

\usepackage{listings}

% \usepackage{listingsutf8} % kannste vergessen, chars müssen in ein byte passen

% replace special characters
\lstset{
  literate={ö}{{\"o}}1
           {Ö}{{\"O}}1
           {ä}{{\"a}}1
           {Ä}{{\"A}}1
           {ü}{{\"u}}1
           {Ü}{{\"U}}1
           {ß}{{\ss}}1
}

\usepackage{framed}
\colorlet{shadecolor}{grau}
\newcommand{\codeline}[1]			{\colorbox{grau}{\texttt{#1}}}
% \newcommand{\codebox}[1]			{\begin{shaded}\texttt{#1}\end{shaded}}
% \newcommand{\codebox}[1]			{\begin{lstlisting}{\texttt{#1}}\end{lstlisting}}
\lstnewenvironment{codebox}[1][] 	{\lstset{		language = #1,
													backgroundcolor = \color{grau},
                                                    % basicstyle = \ttfamily,
                                                    basicstyle = \scriptsize,
                                                    % basicstyle = \tiny,
                                                    numbers = left, % none
                                                    numberstyle = \tiny,
                                                    breaklines = true,
                                                    breakatwhitespace = true,
                                                    escapeinside = {@}{@},
                                                    frame = single,
                                                    tabsize = 2
}}{}
\newcommand{\codefile}[3] 		{\lstinputlisting[	language = #1,
                                                    % extendedchars = true,
                                                    % inputencoding = utf8/utf8,
													backgroundcolor = \color{grau},
                                                    basicstyle = \ttfamily,
                                                    numbers = left, % none
                                                    numberstyle = \tiny,
                                                    breaklines = true,
                                                    breakatwhitespace = true,
                                                    frame = single,
                                                    tabsize = 2,
													% firstline=42
													% lastline=50
													#2]
{#3}}


% JAVASCRIPT SUPPORT
\lstdefinelanguage{JavaScript}{
  keywords={typeof, new, true, false, catch, function, return, null, catch, switch, var, if, in, while, do, else, case, break},
  keywordstyle=\color{blue}\bfseries,
  ndkeywords={class, export, boolean, throw, implements, import, this},
  ndkeywordstyle=\color{darkgray}\bfseries,
  identifierstyle=\color{black},
  sensitive=false,
  comment=[l]{//},
  morecomment=[s]{/*}{*/},
  commentstyle=\color{purple}\ttfamily,
  stringstyle=\color{red}\ttfamily,
  morestring=[b]',
  morestring=[b]"
}
\lstset{
   language=JavaScript,
   backgroundcolor=\color{lightgray},
   extendedchars=true,
   basicstyle=\footnotesize\ttfamily,
   showstringspaces=false,
   showspaces=false,
   numbers=left,
   numberstyle=\footnotesize,
   numbersep=9pt,
   tabsize=2,
   breaklines=true,
   showtabs=false,
   captionpos=b
}






%%%%%%%%%%%%%%%	Mathematik etc. %%%%%%%%%%%%%%%

% \usepackage{MnSymbol}

% conflict with wasysym package
\let\iint\relax
\let\iiint\relax

% \usepackage{amsmath} % not needed with mathtools
\usepackage{amssymb}
% \usepackage{complexity} % not compatible with something

\usepackage{mathtools}


\usepackage{mathcomp}



% cool mathmode font
\usepackage{eulervm}
% ... with global font
\usepackage{mathpazo}


% --- Einheiten: ---
% (Syntax: \SI{wert}{einheit})
% \usepackage{siunitx}
% \sisetup
% {
%  	alsoload=binary,    % Binäre Einheiten (\bit, \byte)
%   % alsoload=synchem,   % Chemische Einheiten (\mmHg, \molar, \torr, ...)
%   % alsoload=astro,     % Astronomische Einheiten (\parsec, \lightyear)
%   % alsoload=hep,       % Einheiten der Hochenergie-Physik (\clight, \eVperc)
%   % alsoload=geophys,   % Einheiten der Geophysik (\gon)
%   % alsoload=chemeng    % Einheiten der chem. Verfahrenstechnik (\gmol, \kgmol, ...)
% 	unitsep=thin, valuesep=space
% }

% \newcommand{\lor}				{\ensuremath{\vee}}
% \newcommand{\land}			{\ensuremath{\wedge}}
\newcommand{\lxor}				{\ensuremath{\oplus}}

\newcommand{\setunion} 			{\ensuremath{\cup}}
\newcommand{\setintersection} 	{\ensuremath{\cap}}
 
\newcommand{\R}					{\ensuremath{\mathbb{R}}}
\newcommand{\N}					{\ensuremath{\mathbb{N}}}
\newcommand{\Z}         {\ensuremath{\mathbb{Z}}}
\newcommand{\Q}					{\ensuremath{\mathbb{Q}}}

\newcommand{\bigO}				{\ensuremath{\mathcal{O}}} 			% big-O notation/symbol

\newcommand{\ComplexityClassP}         {\ensuremath{P}}      % complexity class
\newcommand{\ComplexityClassE}         {\ensuremath{E}}      % complexity class
\newcommand{\ComplexityClassEXP}       {\ensuremath{EXP}}      % complexity class
\newcommand{\ComplexityClassNP}        {\ensuremath{NP}}      % complexity class
\newcommand{\ComplexityClassNE}        {\ensuremath{NE}}      % complexity class
\newcommand{\ComplexityClassNEXP}      {\ensuremath{NEXP}}      % complexity class
\newcommand{\ComplexityClassFP}        {\ensuremath{FP}}      % complexity class

%!TEX root = 0-main.tex

% Author: Philipp Moers <soziflip@gmail.com> 



\newcommand{\datum}[1]
{
    \begin{center}
    \textcolor[HTML]{66CC66}
    {
        \rule{1\textwidth}{0.3pt}\\[6pt]
        Vorlesung vom #1
        \rule{1\textwidth}{0.3pt}\\[6pt]
    }
    \end{center}
}


% \newenvironment{definition}[1][]
\newenvironment{definition}
{
    % \textbf{Definition: #1}

    \underline{\textbf{Definition}}

    % \begin{addmargin}[3][0]
    % \newgeometry{left=3cm,bottom=0.1cm}
    % \begin{quote}
}
{
    % \end{quote}
    % \restoregeometry
    % \end{addmargin}
}


\newcommand{\definiere}[1]
{
    \textbf{#1}
}



\newenvironment{satz}
{

    \underline{\textbf{Satz}}
    
}
{
}

\newenvironment{beweis}
{

    \underline{Beweis}
    
}
{
}





\newenvironment{beispiel}
{
    \begin{quote}
    \underline{\textbf{Beispiel}}
    
}
{
    \end{quote}
}

\begin{document}
\selectlanguage{ngerman}

\begin{titlepage}
    \begin{center}
        \Large{Ludwig-Maximilians-Universität München}\\[1cm]
        \large{\scshape{WS 2015/2016}}\\
        \large{\scshape{Martin Hofmann, Ulrich Schöpp}}\\[3cm]
        \Huge{\textbf{Komplexitätstheorie}}\\[5cm]
        \large{Vorlesungsmitschrieb von}\\[1cm]
        \large{Philipp Moers \\ 
        <p.moers@campus.lmu.de>\\
        <soziflip@gmail.com>}\\[2cm]
        \vfill
        \footnotesize{Stand: \today, \currenttime}
    \end{center}
\end{titlepage}

\begin{abstract}

    Die Komplexitätstheorie beschäftigt sich mit der Klassifikation von Algorithmen und Berechnungsproblemen nach ihrem Ressourcenverbrauch, z.\,B. Rechenzeit oder benötigtem Speicherplatz. Probleme mit gleichartigem Ressourcenverbrauch werden zu Komplexitätsklassen zusammengefasst. Die bekanntesten Komplexitätsklassen sind sicherlich P und NP, die die in polynomieller Zeit deterministisch bzw. nicht-deterministisch lösbaren Probleme umfassen.

    P und NP sind jedoch nur zwei Beispiele von Komplexitätsklassen. Andere Klassen ergeben sich etwa bei der Untersuchung der effizienten Parallelisierbarkeit von Problemen, der Lösbarkeit durch zufallsgesteuerte oder interaktive Algorithmen, der approximativen Lösung von Problemen, um nur einige Beispiele zu nennen.

    
    \begin{center}
    \textbf{Anmerkung}
    \end{center}

    Dies ist ein inoffizieller Vorlesungsmitschrieb. Als solcher erhebt er keinen Anspruch auf (NP-)Vollständigkeit oder Korrektheit. Nutzung, Anmerkungen und Korrekturen sind jedoch durchaus erwünscht! 

    Website der Vorlesung: \url{http://www.tcs.ifi.lmu.de/lehre/ws-2015-16/kompl}
    
\end{abstract}


\tableofcontents


\newpage

%!TEX root = 0-main.tex

% Author: Philipp Moers <soziflip@gmail.com> 



\chapter{Einführung} % (fold)
\label{cha:einfuhrung}



\section{Motivation}

Theoretische Informatik, Berechenbarkeit und insbesondere Komplexitätstheorie ist \emph{der} Informatiker-Shit schlechthin. Let's do it!



\section{Literatur}

Die Vorlesung basiert hauptsächlich auf folgendem Buch:
\begin{itemize}
    \item Bovet, Crescenzi. Introduction to the Theory of Complexity. Prentice Hall. New York. 1994.
\end{itemize}

Weiterhin ist folgende Literatur gegeben:
\begin{itemize}
    \item C. Papadimitriou. Computational Complexity. Addison-Wesley. Reading. 1995.
    \item I. Wegener. Komplexitätstheorie: Grenzen der Effizienz von Algorithmen. Springer. 2003.
    \item S. Arora und B. Barak. Complexity Theory: A Modern Approach. 
\end{itemize}

Zur Motivation:
\begin{itemize}
    \item Heribert Vollmer. Was leistet die Komplexitätstheorie für die Praxis? Informatik Spektrum 22 Heft 5, 1999.
    \item Stephen Cook: The Importance of the P versus NP Question. Journal of the ACM (Vol. 50 No. 1)
\end{itemize}







% chapter einfuhrung (end)

%!TEX root = 0-main.tex

% Author: Philipp Moers <soziflip@gmail.com> 



\datum{12.10.15}

\chapter{Turingmaschinen, Berechenbarkeit und Komplexität} % (fold)
\label{cha:turingmaschinen_berechenbarkeit_und_komplexitaet}



\section{Turingmaschinen}


\begin{definition}
        
    Eine \definiere{Turingmaschine} $T$ mit $k$ Bändern ist ein 5-Tupel
    $$ T = (Q, \Sigma, I, q_0, F) $$
    \begin{itemize}
        \item $Q$ ist eine endliche Menge von Zuständen
        \item $\Sigma$ ist eine endliche Menge von Bandsymbolen, $\square \in \Sigma$
        \item $I$ ist eine Menge von Quintupeln der Form $(q, s, s', m, q')$ mit $q, q' \in Q$ und $s, s' \in \Sigma^k$ und $m \in \{ L, R, S \}^k$ 
        \item $q_0 \in Q$ Startzustand
        \item $F \subseteq Q$ Endzustände
    \end{itemize}

    $\square$ ist das Leerzeichen oder \definiere{Blanksymbol}.

    $T$ heißt \definiere{deterministisch} genau dann, wenn für jedes $q \in Q$ und $s \in \Sigma^k$ genau ein Quintupel der Form $(q, s, \_, \_, \_) \in I$ existiert. Sonst heißt $T$ \definiere{nichtdeterministisch}.

    Eine Turingmaschine heißt \definiere{Akzeptormaschine} genau dann, wenn zwei Zustände $q_A, q_R \in F$ speziell markiert sind. $q_A$ signalisiert Akzeptanz, $q_R$ signalisiert Verwerfen der Eingabe.

    Eine Turingmaschine heißt \definiere{Transducermaschine} genau dann, wenn ein zusätzliches Band ausgezeichnet ist (das Ausgabeband).

\end{definition}



\begin{beispiel}
\label{nullenundeinsen}

    Akzeptormaschine $T$ für Sprache $L = \{ 0^n 1^n | n \geq 0 \}$ wobei $\Sigma= \{0,1\}, Q = \{q_0, \dots q_4 \}$

    $T$ wird deterministisch sein. $T = (Q, \Sigma, I, q_0, F), q_A = q_1, q_R = q_2, F = \{q_1, q_2\}, k=2$

    \vspace{2pt}
    \begin{tabular}{|c|c|c|c|c|c|c|c|}\hline
    \rowcolor{grau} $q$   & $s_1$     & $s_2$     & $s_1'$    & $s_2'$    & $m_1$ & $m_2$ & $q'$    \\\hline
                    $q_0$ & $\square$ & $\square$ & $\square$ & $\square$ & $S$   & $S$   & $q_1$   \\\hline
                    $q_0$ & $0$       & $\square$ & $0$       & $0$       & $R$   & $R$   & $q_3$   \\\hline
                    $q_0$ & $1$       & $\square$ & $1$       & $\square$ & $S$   & $S$   & $q_2$   \\\hline
                    $q_3$ & $\square$ & $\square$ & $\_$      & $\_$      & $\_$  & $\_$  & $q_2$   \\\hline
                    $q_3$ & $0$       & $\square$ & $0$       & $0$       & $R$   & $R$   & $q_3$   \\\hline
                    $q_3$ & $1$       & $\square$ & $1$       & $\square$ & $S$   & $L$   & $q_4$   \\\hline
                    $q_4$ & $0$       & $0$       & $\_$      & $\_$      & $\_$  & $\_$  & $q_2$   \\\hline
                    $q_4$ & $1$       & $0$       & $1$       & $0$       & $R$   & $L$   & $q_4$   \\\hline
                    $q_4$ & $0$       & $\square$ & $\square$ & $\square$ & $S$   & $S$   & $q_1$   \\\hline
                    $\_$  & $\_$      & $\_$      & $\_$      & $\_$      & $\_$  & $\_$  & $q_2$   \\\hline
    \end{tabular}

\end{beispiel}


Die \definiere{globale Konfiguration} (oder der \definiere{Zustand}) einer Turingmaschine beinhaltet die Beschriftung aller Bänder, den internen Zustand ($\in Q$) und die Positionen aller $k$ Lese-/Schreibköpfe. Globale Konfigurationen können als endliche Wörter über einem geeigneten Alphabet (z.\,B. $\{0,1\}$) codiert werden.


Eine Turingmaschine \definiere{akzeptiert} eine Eingabe genau dann, wenn eine Berechnungsfolge ausgehend von dieser Eingabe existiert und in einem Zustand aus $F$ endet.

Eine Turingmaschine \definiere{akzeptiert} eine Sprache $L \subseteq \left( \Sigma \setminus \{\square\} \right)^\ast$ falls gilt: 
$$ \text{Die Turingmaschine akzeptiert } w \Leftrightarrow w \in L $$

Eine Turingmaschine \definiere{entscheidet} eine Sprache $L \subseteq \left( \Sigma \setminus \{\square\} \right)^\ast$ genau dann, wenn sie sie akzeptiert und eine/die Berechnung in $q_A$ endet.













\datum{15.10.15}


Zu \definiere{Mehrband-Turingmaschinen}:

Bisher waren die Bänder beidseitig unendlich. Ab jetzt und im Buch sind sie nur noch einseitig unendlich.



\begin{satz}
    Eine Mehrband-Turingmaschine mit $k$ Bändern kann durch eine Einband-Turingmaschine simuliert werden.

    Dies benötigt quadratischen Mehraufwand.
\end{satz}

\begin{beweis}
    Die Beweisidee nutzt für das alte Alphabet $\Sigma$ das neue Alphabet $\Sigma^k \times \{0,1\}^k$, das die Zeichen auf den Bändern und, ob der Lese-/Schreibkopf an dieser Position steht, speichert.
\end{beweis}



\begin{definition}
    Eine \definiere{universelle Turingmaschine} erhält als Eingabe $(M, x)$, wobei $M$ die Beschreibung einer Turingmaschine in geeignetem Binärformat und $x$ die Eingabe für $M$ ist.
    Sie berechnet dann die Ausführung von $M$ auf $x$.
\end{definition}




\section{Halteproblem}


\begin{definition}
    Gegeben Turingmaschine und Eingabe $(M, x)$. Das Problem, zu entscheiden, ob $M$ angewendet auf $x$ hält oder nicht, heißt \definiere{Halteproblem}.
\end{definition}

\begin{satz}
    Das Halteproblem ist unentscheidbar.
\end{satz}

\begin{beweis}
    Angenommen, es gäbe eine Turingmaschine $M_{HALT}$, die das Halteproblem entscheidet.

    Dann könnten wir auch eine neue Turingmaschine $M_D$ konstruieren:\\
    Simuliere Eingabe $M$ auf $M$ selbst und schaue, ob sie hält. Falls ja, dann gehe in Endlosschleife. Falls nicht, halte an.

    Für $M = M_D$ ergibt sich nun ein Widerspruch: Falls sie hält, hält sie nicht. Falls sie nicht hält, hält sie.
\end{beweis}




\section{Rechenzeit}


\begin{definition}
    Die \definiere{Rechenzeit} definiert man wie folgt:

    Gegeben eine Turingmaschine $M$ und Eingabe $x$. 

    $TIME_M(x)$ ist die Dauer (Anzahl der Schritte) der Berechnung von $M$ auf $x$.
\end{definition}

% Im Beispiel \ref{nullenundeinsen} bla
Im Beispiel der Maschine für $L = \{ 0^n 1^n | n \geq 0 \}$ ist $TIME_M(x) = |x| $ (Länge des Strings).




\begin{satz}
    Das \definiere{Speedup-Theorem} besagt, dass zu jeder Turingmaschine $M$ eine äquivalente Turingmaschine $M'$ konstruiert werden kann, sodass

    $$ TIME_{M'}(x) \leq \frac{1}{k} * TIME_M(x) $$

    wobei $k \in \N \setminus \{0\}$ fest gewählt ist.
\end{satz}

Zum Beispiel ist bei $k = 7$ die neue Turingmaschine siebenmal so schnell.

\begin{beweis}
    Gegeben $M$ mit Alphabet $\Sigma$.

    Dann wird $M'$ mit Alphabet $\Sigma^k$ konstruiert. Ein Symbol von $M'$ repräsentiert $k$ aufeinanderfolgende Symbole von $M$, d.h. $M'$ kann $k$ Schritte von $M$ ein einem einzigen ausführen.
\end{beweis}

\textbf{Anmerkung:}

Die Schritte werden in der Praxis also schon aufwändiger, die definierte Metrik $TIME$ erfasst das nur nicht. No Magic here.



\begin{definition}
    Sei $f: \N \rightarrow \N$. 

    Dann definieren wir $DTIME(f)$ als Menge aller Entscheidungsprobleme (oder Berechnungsprobleme) $A$, zu denen eine \underline{deterministische} Turingmaschine $M$ existiert, sodass $M$ $A$ entscheidet und die Rechenzeit in $\bigO(f(n))$ liegt.

    $$ DTIME(f) = \{  
        A 
        \ |\ 
        \exists M: 
            M \text{ entscheidet } A
            \text{ und }
            \forall x \in \Sigma^\ast: TIME_M(x) = \bigO(f(|x))
    \}
    $$
\end{definition}

\begin{satz}
    Matrixmultiplikation liegt in $\bigO(n^3)$, also in $DTIME(\sqrt{n}^3)$, wenn die Länge der Matrix auf dem Band $n$ ist.
    Sie liegt sogar in $\bigO(n^{2.78})$
\end{satz}

Offen ist die Frage, ob sie in $DTIME(\sqrt{n}^2)$ liegt.



\section{Komplexitätsklassen}


\begin{definition}
    Eine Menge der Form $DTIME(f(n))$ heißt \definiere{deterministische Zeitkomplexitätsklasse}.
    Analog heißt $NTIME(f(n))$ für nichtdeterministische Turingmaschinen \definiere{nichtdeterministische Zeitkomplexitätsklasse}.
\end{definition}


Wir betrachten zu gegebener Funktion $f: \N \rightarrow \N$ durch Turingmaschine $M$ folgenden Algorithmus:

Rechne $M$ auf Eingabe $M$ selbst für $f(M)$ Schritte. Falls $M$ sich bis dahin akzeptiert, verwerfe die Eingabe. Falls sie sich verwirft oder bis dahin nicht gehalten hat, akzeptiere die Eingabe.

Das durch diesen Algorithmus beschriebene Problem 
$$K_f = \{ M \ |\ M \text{ akzeptiert sich selbst nicht in höchstens } f(|M|) \text{ Schritten.} \}$$ 
ist ``offensichtlich'' entscheidbar. Die Rechenzeit für diese Entscheidung muss aber im allgemeinen $f(|M|)$ übersteigen.
    
Wäre $M_f$ eine Turingmaschine, die $K_f$ entscheidet und außerdem $TIME_{M_f} \leq f(|x|)$ für alle $x$, dann führt die Anwendung von $M_f$ auf $M_f$ selbst zum Widerspruch (wie beim Halteproblem).

$f$ muss dazu selbst in Zeit $\bigO(f(n))$ berechenbar und monoton steigend sein. Man nennt $f$ dann \definiere{zeitkonstruierbar}.

Durch geschickte Ausnutzung dieses Arguments erhält man den \definiere{Zeit-Hierarchie-Satz}:

\begin{satz}
    Falls $f: \N \rightarrow \N$ zeitkonstruierbar ist, dann gilt:
    $$ DTIME(f(n)) \subset DTIME(f(n)*\log^2(f(n))) $$
    wobei $\subset$ eine echte Teilmengenbeziehung bezeichnet.
\end{satz}

\textbf{Anmerkung:}

``Vernünftige'' Funktionen wie $2^n$, $\log(n)$, $\sqrt{n}$ etc. sind zeitkonstruierbar.




\begin{satz}
    Nach Borodin und Trakhtenbrot gilt das \definiere{Gap-Theorem}:

    Für eine totale, berechenbare Funktion $g: \N \rightarrow \N$ mit $g(n) \geq n$ gibt es immer eine totale, berechenbare Funktion $f: \N \rightarrow \N$, sodass gilt:
    $$ DTIME(f) = DTIME (g \circ f) $$

    Es gibt also in der Hierarchie der Komplexitätsklassen beliebig große Lücken.
\end{satz}


\begin{definition}
    \definiere{Wichtige Komplexitätsklassen:} 
    $$ \ComplexityClassP = \bigcup_{k \geq 1} DTIME(n^k) $$
    $$ \ComplexityClassE = \bigcup_{k \geq 1} DTIME(2^{kn}) $$
    $$ \ComplexityClassEXP = \bigcup_{k \geq 1} DTIME(2^{n^k}) $$
\end{definition}

Nach dem \definiere{Zeit-Hierarchie-Satz} gilt:
$$ \ComplexityClassP \subset \ComplexityClassE \subset \ComplexityClassEXP $$













\datum{19.10.15}

\begin{definition}

    \definiere{Nichtdeterministische Zeitkomplexität}
    Sei $T$ eine nichtdeterministische Turingmaschine. 

    Für $x \in \Sigma$ ist $NTIME_T(x)$
    \begin{enumerate}
        \item definiert genau dann, wenn alle Berechnungen von $T$ auf $x$ halten
        \item Falls definiert und $x \in L(T)$ (d.h. es gibt eine akzeptierende Berechnung von $T$ auf $x$) definiert als Länge der kürzesten akzeptierenden Berechnungen von $T$ auf $x$.
        \item Falls überhaupt definiert $x \notin L(T)$ so ist $NTIME_{(x)}$ die Länge der kürzesten Berechnung.
    \end{enumerate}

\end{definition}


\begin{definition}
    % TODO esox ? was ist das?
    % TODO verstehen
    Nichtdeterministische Komplexitätsklassen 
    $$ NTIME(f(n)) = \{ L | \exists T \text{ mit } L(T)=L \text{ und } NTIME_T(x) = \bigO(f(|x|)) \} $$

    Es gibt einen nichtdeterministischen Zeithierachiesatz.

    % P=U DTIME(n^k), NP = U NTIME(n^k)
    % k>=1                k>=1

    % E = U DTIME(2^kn), NE= U NTIME(2^KN)
    % k>=1               k>=1

    % EXP = DTIME(2^n^k), NEXP= U NTIME(2^N^k)
    % k>=1                k>=1

    $$ \ComplexityClassNP   = \bigcup_{k \geq 1} NTIME(n^k) $$
    $$ \ComplexityClassNE   = \bigcup_{k \geq 1} NTIME(2^{kn}) $$
    $$ \ComplexityClassNEXP = \bigcup_{k \geq 1} NTIME(2^{n^k}) $$

\end{definition}


Nichtdeterminismus kann durch erschöpfende Suche deterministsch simuliert werden.
z.B. $\ComplexityClassNP \subseteq \ComplexityClassEXP$


Allgemein:
$$NTIME(f(n)) \subseteq DTIME(2^{\bigO(f(n)})$$

(zeitkonstruierbar)
% ........\ --- Zeitkonstruierbar


\section{polynomielle Verifizierbarkeit}


\begin{definition}
    
    Charakterisierung von $\ComplexityClassNP$ durch \definiere{polynomielle Verifizierbarkeit} $PV$.

    % TODO z h p? da stimmt doch was nicht
    $$ L \subseteq \Sigma^\ast . L \in PV \text{ genau dann, wenn } $$
    $$ \exists L' \in P \text { sodass gilt } x \in L \Leftrightarrow  \exists z \text{ ``Loesung'' mit } |h| \leq L' $$
    $$ \text{ wobei } p \text{ ein Polynom ist } $$

\end{definition}


\begin{satz}
    $$ \ComplexityClassNP = PV $$
\end{satz}
\begin{beweis}
    ``$\subseteq$'':

    $L \in \ComplexityClassNP$ . Sei $T$ eine nichtdeterministische Turingmaschine für $L$ mit Laufzeit $p(n)$.

    $$ L' = \{(x,y) | y \text{ codiert eine akzeptierende Berechnung von } T \text{ auf } x \} $$

    \begin{enumerate}
        % TODO . (x,y,) ??? dafuq clemens
        \item $x \in L \Leftrightarrow  \exists y: |y| \leq (p(|x|))^2 . (x,y,) \in L'$
        \item $L' \in \ComplexityClassP$
    \end{enumerate}


    ``$\supseteq$'':

     Gegeben $L, L' \in P$. 

     Eine nichtdeterministische Turingmaschine $T$ für $L$ rät zunächst $y$ und prüft dann $(x,y) \in L'$
\end{beweis}
  

$\ComplexityClassEXP$ im Gegensartz zu $\ComplexityClassNP$ bzw. $PV$ umfasst auch Probleme mit exponentiell großen Lösungen bzw solchen wo die Verifikation einer Lösung einen exponentiellen Aufwand macht.






% chapter turingmaschinen_berechenbarkeit_und_komplexitaet (end)

%!TEX root = 0-main.tex

% Author: Philipp Moers <soziflip@gmail.com> 




\chapter{NP und P} % (fold)
\label{cha:np_und_p}
        

\section{Padding}

\begin{definition}
    Sei $L \subseteq \Sigma^\ast$ eine Sprache.

    % (Blase: 2^?  ist die Paddingfunktion geht auch für andere P.fkt.)
    % TODO verstehen, anpassen, einkommentieren

    $$ padd(L) = \{1^l O x | x \in L, l = 2^{|x|} \} $$
\end{definition}


\begin{satz}
    
    Es gilt: 
    $ L \in DTIME(f(s^n)) $, dann ist
    $$ padd(L) \in DTIME(f(n)) $$
    Blase $f$ zeitkonstant und insbesondere $f(n) \geq n$

\end{satz}

\begin{beweis}
    
    Sei $T$ eine deterministische Turingmaschine für $L$ und
    % TODO \leq \in ?
    $DTIME_T(x) \leq \in x f(2^n)$

    Die folgende Maschine $T'$ entscheided $padd(L)$:

    Gegeben Eingabe $y$, schreibe $y + 1^l O x$ und prüfe ob $l = 2^{|x|}$
    geht in Zeit $\bigO(|y|)$

    Aufwand: 
    $c f(2^{|x|}) \leq c * f(|y|)$

    Gesamtaufwand:
    $\leq c * f(|y|) + |y|
    = \bigO(f(|y|))$

    falls $f(n) \geq n$ (zeitkonstruierbar)
\end{beweis}




\begin{satz}
    Umgekehrt gilt auch:

    Wenn $padd(L) \in DTIME(f(n))$
    dann $L \in DTIME(f(s^{n+1}))$
\end{satz}

\begin{beweis}
    Sei $T$ eine Turingmaschine für $padd(L)$
    mit $DTIME_T(y) \leq c f(|y|)$

    Wir bauen eine Maschine für $L$:
    Gegeben Eingabe $x$, bilde $y = 1^{2^{|x|}} O x $.

    Aufwand: $\bigO(2^{|x|})$
    Setze $T$ auf $y$ an.
    Aufwand: $c * f(|y|) = c * f(2^{|x|} + |x| + 1)$

    Gesamtaufwand:
    $\bigO(f(2^{|x|+1}))$
\end{beweis}




\section{Was wenn P = NP}

\begin{satz}
    Folgerung:
    $$\ComplexityClassP = \ComplexityClassNP \Rightarrow \ComplexityClassE = \ComplexityClassNE$$
\end{satz}
\begin{beweis}
    Sei $P = NP$
    und $L \in NE$ und $T$ eine Maschine mit Aufwand $n^{kk}$
    wobei $k$ fest, $n$ Länge der Eingabe.
    
    $L \in NTIME(n^{nk}) = NTIME((2^n)^k)$
    Also $padd(L) \in NTIME(2^k)$
    also $padd(L) \in  NP$ und nach Annahme $padd(L) \in P$

    Also $padd(L) \in DTIME(n^{k'})$
    also $L \in DTIME((2^{n+1})^{k'})
    = DTIME(2^{k'n + k'}) = DTIME(2^{k'n}) \subseteq E$
\end{beweis}



Slogan: Gleichheit von Komplexitätsklassen vererbt sich nach oben.

Mit anderen Paddingfunktion zeigt man ebenso:

$$ \ComplexityClassP = \ComplexityClassNP \Rightarrow \ComplexityClassEXP = \ComplexityClassNEXP $$
$$ \ComplexityClassE = \ComplexityClassNE \Rightarrow \ComplexityClassEXP = \ComplexityClassNEXP $$

Kontrapositiv ausgedrückt:

$$ \ComplexityClassE \neq \ComplexityClassNE \Rightarrow \ComplexityClassP \neq \ComplexityClassNP $$ etc.


Es koennte sein, dass $\ComplexityClassP \neq \ComplexityClassNP$ aber doch $\ComplexityClassE = \ComplexityClassNE$.

Slogan: Trennung von Komplexitätklassen vererbt sich (durch Padding) von oben nach unten.




\section{Sogenannte Effizienz}


$\ComplexityClassP$ wird gemeinhin gleichgesetzt mit ``effizient loesbar''.
    
Wachstumsverhalten:

  $p$ Polynom. $\Rightarrow$ 
  $$ \exists c > 0:   p(2n) \leq c * p(n) $$

  Bei Verdopplung des Inputs wird der Output also ver-$c$-facht.

  Häufig hat eine Bruteforce-Lösung (``stures Durchprobieren'')
  exponentiellen und eine echte algorithmische Lösung hat polynomiellen Aufwand.











\datum{09.11.15}


\section{Polynomialzeitreduktionen}



\begin{definition}

    $f : \Sigma^\ast \rightarrow \Sigma^\ast $
    beziehungweise für Binärkodierung
    $f : \N \rightarrow \N $

    ist in \definiere{$\ComplexityClassFP$} (eine \definiere{in Polynomialzeit berechenbare Funktion}) genau dann, wenn eine polynomialzeitbeschränkte (deterministische) Transducer-Maschine existiert, die $f$ berechnet.

\end{definition}


Gegeben Input $x \in \Sigma^\ast$, dann hält die Maschine nach $\leq p(|x|)$ Schritten mit Ergebnis $f(x)$, wobei $p$ ein Polynom ist.

\begin{beispiel}
    
    \textbf{Musterbeispiele:}

    \begin{itemize}
        \item Alle Polynome sind in $\ComplexityClassFP$, zum Beispiel $f(x) = x^3 + 10x^2 + x$
        \item Charakteristische Funktionen aller Probleme in $\ComplexityClassP$ sind in $\ComplexityClassFP$.
        \item Matrixmultiplikation ist in $\ComplexityClassFP$.
    \end{itemize}

    \textbf{Gegenbeispiele:}

    \begin{itemize}
        \item $f(x) = 2^x$ ist nicht in $\ComplexityClassFP$, denn $|2^x| = x + 1 \geq 2^{|x| + 1} +  1 = \Omega(2^{|x|})$
        \item Charakteristische Funktionen von Problemen in $\ComplexityClassEXP \setminus \ComplexityClassP$  sind nicht in $\ComplexityClassFP$.
    \end{itemize}

    \textbf{Wahrscheinliches Gegenbeispiel:}

    $f(n) = \text{größter Teiler von } n \text{ außer } n \text{ selbst.}$
    Algorithmus: Alle Zahlen von 1 bis $n$ durchlaufen und testen, immer merken, wenn größerer Teiler gefunden.\\
    Laufzeit: $\Omega(n) = \Omega(2^{|n|})$ (Länge der Eingabe statt Zahl selbst)

\end{beispiel}




\begin{satz}
    $\ComplexityClassFP$ ist unter Komposition (Hintereinanderausführung) abgeschlossen.\\
\end{satz}
\begin{beweis}
    
    Seien $ f: \Sigma^\ast \rightarrow \Sigma^\ast, g : \Sigma^\ast \rightarrow \Sigma^\ast \in \ComplexityClassFP $

    Wir definieren $h(x) = g(f(x))$

    Programm für $h$: $y = f(x);\ z = g(y);\ return z;$\\
    Gesamtaufwand: $O(p_f(|x|) + p_g(p_f(|x|))$, wobei $p_f$ und $p_g$ Polynome sind, die die Laufzeit für Algorithmen für $f$ bzw. $g$ beschränken.
\end{beweis}




\begin{definition}

    \definiere{Polynomielle Reduzierbarkeit}

    Seien $L_1, L_2a \subseteq \Sigma^\ast$.

    Wir sagen $L_1$ ist polynomiell auf $L_2$ reduzierbar (\definiere{$\ComplexityClassFP$-reduzierbar}) genau dann, wenn $f \in \ComplexityClassFP$ existiert, sodass $x \in L_1 \Leftrightarrow f(x) \in L_2$.

    Aus Algorithmus für $L_2$ erhält man einen für $L_1$, indem man $f(x)$ berechnet und prüft, ob das Ergebnis in $L_2$ liegt.

    In Zeichen: $L_1 \leq_p L_2$ oder $L_1 \leq L_2$, auch $f: L_1 \leq L_2$

\end{definition}


\begin{beispiel}
    
    3COL = $\{ G = (V,E)\ |\ G \text{ kann mit 3 Farben gefärbt werden, d.h. } \exists c: V \rightarrow \{r,g,b\} \text{ sodass } \forall (u, u') \in E: c(v) \neq c(u') \}$

    SAT = $\{ \phi\ |\ \phi \text{ aussagenlogische Formel die erfüllbar ist} \}$


    Behauptung: 3COL $\leq$ SAT



    % $f((V,E)) = (\bigwedge \bigvee         x_{v,y}  )      \land      \bigwedge   \bigwedge    \bigwedge      x_{v,c} \rightarrow \neg x_{v,c'}$
    %             v \in V    y \in r,g,b                               v \in V      c \in r,g,b    c' \in r,g,b
    %                                                                                               c \neq c'

    %             $ \land    \bigwedge        \bigwedge         x_{v,y} \rightarrow \neg x_{v', y}$
    %                        (v,v') \in E     y \in {r,g,b}


    $$f((V,E)) = \left(     \bigwedge_{v \in V} \bigvee_{y \in \{r,g,b\}}    x_{v,y}    \right)         \land     $$ 

    $$ \left(     \bigwedge_{v \in V} \bigwedge_{c \in \{r,g,b\}} \bigwedge_{c' \in \{r,g,b\}}      x_{v,c} \rightarrow \neg x_{v,c'} \right)       \land     $$

    $$ \left(     \bigwedge_{(v,v') \in E} \bigwedge_{y \in \{r,g,b\}}    x_{v,y} \rightarrow \neg x_{v', y} \right) $$
                                


    Offensichtlich ist $f \in \ComplexityClassFP$ und $G \in \text{ 3COL } \Leftrightarrow f(G) \in \text{ SAT }$,\\ also 3COL $\leq$ SAT.

\end{beispiel}



\begin{beispiel}
    
    KNFSAT := wie SAT, aber auf konjunktive Normalform eingeschränkt.

    Es gilt trivialerweise KNFSAT $\leq$ SAT, aber auch SAT $\leq$ KNFSAT (Durch Einführung von Abkürzungen von Teilformeln).

\end{beispiel}

\begin{beispiel}
    
    3SAT := KNFSAT eingeschränkt auf Klauseln mit 3 Literalen.

    Es gilt KNFSAT $\leq$ 3SAT.

    Man kennt keine Reduktion von 3SAT auf 2SAT.

\end{beispiel}


\begin{beispiel}
    
    NODE-COVER := $\{ G = (V,E), n\ |\ \exists U \subseteq V: |U| \leq n \text{ und } \forall (v,v') \in E: v \in U \lor v' \in U \}$

    Es gilt NODE-COVER $\leq$ KNFSAT.

    und - was schwieriger zu zeigen ist - KNFSAT $\leq$ NODE-COVER:

    Gegeben: KNF $\phi$ mit $m$ Variablen $x_1 \dots x_m$ und $k$ Klauseln $C_1 \dots C_k$ wobei $C_j = l_{j,1} \lor \dots \lor l_{j,k}$\\
    Falls 3SAT, so sind alle $k_j = 3$.\\
    Die $l_{j,i}$ sind Literale, d.h. negierte oder nicht-negierte Variablen.

    Wir konstruieren Graphen $G = (V,E)$ wie folgt:
    \begin{itemize}
        \item Für jede Variable $x_t$ zwei Knoten $x_t, \neg x_t$
        \item Für jede Klausel $C_j$ $k_j$ Knoten $(l_{j,1}) \dots (l_{j,k})$,\\ Insgesamt $2m \Sigma_{j=1}^k k_j$ Knoten
        \item Kanten: 
        \begin{itemize}
            \item $(x_t, \neg x_t)$
            \item Vollständiger Graph für $l_{j,1} \text{ bis } l_{j,k}$
            \item $(x_t, l_{j,i})$ bzw. $(\neg x_t, l_{j,i})$, falls $l_{j,i} = x_t$ bzw. $l_{j,i} = \neg x_t$
        \end{itemize}
    \end{itemize}

\end{beispiel}





\section{NP-Härte und NP-Vollständigkeit}


\begin{definition}
    
    $L \subseteq \Sigma^\ast$ ist \definiere{$\ComplexityClassNP$-hart} ($\ComplexityClassNP$-schwer, $\ComplexityClassNP$-schwierig) genau dann, wenn 

    $$ \forall L' \in \ComplexityClassNP: L' \leq_p L $$

\end{definition}


\begin{satz}
    HALT (Halteproblem) ist NP-hart.
\end{satz}
\begin{beweis}
    Gegeben: $L' \in \ComplexityClassNP$. Baue deterministische Turingmaschine $M$, sodass $M(x)$ hält genau dann, wenn $x \in L'$ Brute-force Suche, Laufzeit exponentiell.\\
    $x \in L' \Leftrightarrow (M, x) \in \text{ HALT } $\\
    also ist $f(x) = (M,x)$ eine Reduktion von $L'$ auf HALT $f: L' \leq \text{ HALT }$.
\end{beweis}



\begin{definition}

    $L \subseteq \Sigma^\ast$ ist \definiere{NP-vollständig} genau dann, wenn $L$ NP-hart ist und $L \in \ComplexityClassNP$.

\end{definition}


\begin{satz}
    HALT $\notin \ComplexityClassNP$.
\end{satz}



\begin{satz}
    \definiere{Satz von Cook}

    SAT ist NP-vollständig.
\end{satz}

\begin{beweis}
    
    SAT $\in \ComplexityClassNP$: trivial.

    Sei $L \in \ComplexityClassNP$ gegeben und o.B.d.A $M$ eine nichtdeterministische Turingmaschine für $L$ mit einem Band $M = (\Sigma, Q, q_0, F, I)$ und $p$ ein Polynom, das die Laufzeit von $M$ beschränkt.
    Gegeben weiterhin $x = x_1 \dots x_n$ Input.

    Gesucht: aussagenlogische Formel $\phi = f(x)$, sodass $\phi$ erfüllbar ist genau dann, wenn $M$ akzeptiert $x$. $q$ muss aus $x$ in polynomieller Zeit berechenbar sein, d.h. $f \in \ComplexityClassFP$.

    $M$ akzeptiert $x$ genau dann, wenn eine akzeptierende Berechnung von $M$ auf $x$ existiert. Solch eine Berechnung hat höchstens $p(n)$ Schritte und o.B.d.A genau $p(n)$ Schritte. Die Bandbeschriftung zu jedem dieser $p(n)$ Schritte besteht aus höchstens $p(n)$ Symbolen und o.B.d.A genau $p(n)$ Symbolen.

    Die Formel $\phi$ verwendet die Variablen 
    \begin{itemize}
        \item $Q_t^i$: Zur Zeit $t$ ist $M$ im Zustand $i$.
        \item $P_{s,t}^i$: Zur Zeit $t$ enthält Bandposition $s$ das $i.$ Symbol.
        \item $S_{s,t}$: Zur Zeit $t$ ist der Kopf in Position $s$.
    \end{itemize}

    $$ q = A \land B \land C \land D \land E \land F $$

    \textit{Details im Buch\dots}

\end{beweis}


3SAT, NODE-COVER sind auch NP-vollständig.

Allgemein gilt: $L$ NP-vollständig und $L' \in \ComplexityClassNP, L \leq L'$ 
so folgt $L'$ NP-vollständig.


\textbf{Anmerkung:}
$\leq$ ist transitiv, da $\ComplexityClassFP$ unter Komposition abgeschlossen ist.

\textbf{Anmerkung:}
3COL, TRAVELINGSALESMAN, SUBSETSUM etc. sind auch NP-vollständig.












\datum{12.11.15}


\section{Etwas zwischen P und NP}



\begin{satz}
    
    \definiere{Satz von Ladner}

    Falls $\ComplexityClassP \neq \ComplexityClassNP$, dann

    $$ \exists A \in \ComplexityClassNP \setminus \ComplexityClassP: A \text{ nicht NP-vollständig.} $$

    $A$ liegt also ``echt'' zwischen $\ComplexityClassP$ und NP-vollständig.

    
\end{satz}



\begin{definition}

    \definiere{Diagonalisierung}
    
    Um zu zeigen, dass eine Sprache $A$ nicht in einer Klasse $\mathcal{C}$ ist, beziehungweise um solch ein $A$ zu konstruieren, kann man eine effektive (FP) Aufzählung von Turingmaschine $(M_i)_i$ verwenden, sodass $\mathcal{C} = \{ L(M_i)\ |\ i \geq 0 \}$  und dann dafür sorgen, beziehungweise zeigen, dass $\forall i: A \neq L(M_i)$ beziehungweise $\forall i: A \triangle L(M_i) \neq 0$.\\
    Das heißt $\forall i \exists x: \left(x \in A \land x \notin L(M_i) \right) \lor \left( x \notin A \land x \in L(M_i) \right)$

\end{definition}


\begin{lemma}
    
    Es existiert eine FP-Funktion $i \mapsto M_i$, sodass $DTIME_{M_i}(x) \leq (|x| + 2)^2$ und $P = \{ L(M_i)\ |\ i \geq 0 \}$.

\end{lemma}

\begin{lemma}
    
    Es existiert eine FP-Funktion $i \mapsto f_i$ wobei $f_i$ eine Übersetzermaschine ist und $FP = \{ f_i\ |\ i \geq 0 \}$ und $DTIME_{f_i}(x) \leq (|x| + 2)^i$. Insbesondere $|f_i(x)| \leq (|x| + 2)^i$.

\end{lemma}

Es ist klar, dass $A \in \ComplexityClassNP$ aber $A \notin \ComplexityClassP$ und $A$ nicht NP-vollständig, wenn
\begin{itemize}
    \item $A \in NP$
    \item $\forall i \exists x: x \in A \triangle L(M_i)$
    \item $\forall i \exists x: x \in SAT \land f_i(x) \notin A \text{ oder } x \notin SAT \land f_i(x) \in A$
\end{itemize}

Das heißt $f_i$ ist keine Reduktion von $SAT$ auf $A$.





Wir konstruieren $A$ in der folgenden Form:

$$ A = \{  x \ |\   x \in SAT  \land   f(|x|) \text{ gerade.}  \} $$

$f$ wird sogleich rekursiv definiert derart, dass dieses $A$ die Bedingungen 1, 2 und 3 erfüllt.

Man sollte also versuchen sicherzustellen, dass 

\begin{itemize}
    \item 
    $f(n) in ZEit p(n)$ berechenbar für Polynom $p$ (Bedingung 1).

    \item 
    Für alle $i$ existiert $x$ mit \\
    $x \in SAT$ und $f(|x|)$ gerade und $x \notin L(M_i)$\\
    oder\\
    ($ x \notin SAT$ oder $f(|x|)$ ungerade ) und $x \in L(M_i)$\\

    \item 
    Für alle $i$ existiert $x$, sodass
    $x \in SAT$ und ($f(|f_i(x)|)$ ungerade oder $f_i(x) \notin SAT$\\
    oder\\
    $x \notin SAT$ und $f(|f_i(x)|)$ gerade und $f_i(x) \in SAT$\\

\end{itemize}


$f$ wird jetzt rekursiv definiert.\\
Wir schreiben $A_f = \{ x \ |\   x \in SAT  \land  f_(|x| \text{ gerade.}  \}$

\begin{equation*}
\begin{split}
f(n+1) = 
    IF    &\ \ (2 + \log \log n)^{f(n)} \geq \log n \\
    THEN  &\ \ f(n) \\
    ELIF  &\ \ \exists x: |x| \leq \log \log n \text{ und } x \in L(M_i) \land x \notin A_f \text{ oder } x \notin L(M_i) \land x \in A_f \\
    THEN  &\ \ f(n) + 1 \text{ ebe } f(n) \\
    ELIF  &\ \ \exists x: |x| \leq \log \log n \\
    THEN  &\ \ f(n) + 1
\end{split}
\end{equation*}

Um $f(n+1)$ zu berechnen wird rekursiv nur auf Werte $f(m)$ mit $m \leq n$ zugegriffen, also ist $f$ eine totale Funktion.

Es genügt, ein Polynom $p(n)$ zu finden, sodass in Zeit $p(n)$ der Wert $f(n+1)$ aus $f(0), f(1), \dots f(n)$ bestimmt werden kann.\\
Die Laufzeit für $f(n)$ ist nämlich dann $\bigO(\Sigma_{m < n} p(m)) = poly(n)$.


Klar ist, dass $A = A_f \neq L(M_i)$ falls $f(n) = 2i+1$ für ein $n$, denn dann war $f(n') = 2i$ für ein $n' < n$ und $f(n'+1) = 2i+1$ also die Suche in Fall 2 erfolgreich.

Ebenso ist $f_i$ keine Reduktion: $SAT \leq A_f$ falls $f(n) = (2i+1)+1$ für ein $n$.

Das heißt wir müssen zeigen, dass $f$ surjektiv ist, d.h. dass jeder Fall irgendwann erfolgreich abgeschlossen wird.

% Für festes $d$ wächst $(2 + \log n)^d$ langsamer als $n$, also wächst auch $ $ blabla

\textit{Details dazu auf der Website.}










\datum{09.11.15}


\section{Orakel-Turingmaschinen}


Orakel-Turingmaschinen als Mittel zu zeigen, dass Beweismethoden ``Diagonalisierung''\footnote{zum Beispiel benutzt für $\ComplexityClassNP \subseteq \ComplexityClassEXP$} und ``Simulation''\footnote{zum Beispiel benutzt für $\ComplexityClassP \subset \ComplexityClassEXP$ (Ladner)} nicht helfen, um $\ComplexityClassP = \ComplexityClassNP$ zu entscheiden.



\begin{definition}
    
    Eine \definiere{Orakel-Turingmaschine} $T$ hat ein zusätzliches Band (Orakelband) und drei zusätzliche Zustände $q_Q$ (Frage), $q_{yes} q_{no}$ (Antwort).

    Ist $A \subseteq \Sigma^\ast$, dann definiert man Berechnungen $T^A(x)$ von $T$ auf $x$ mit Orakel $A$ wie folgt:
    \begin{itemize}
        \item Wie üblich mit der zusätzlichen Regel:\\
            Falls $T$ in $q_Q$ so wird $T$ in Zustand $q_{yes}$, $q_{no}$ versetzt und zwar in einem Schritt, je nach dem, ob die aktuelle Beschriftung $z \in \Sigma^\ast$ des Orakelbands (``Anfrage''/``Query'') in $A$ ist ($q_{yes}$) oder nicht ($q_{no}$).
    \end{itemize}
    Man schreibt $L^A(T)$ oder $L(T^A)$ für die von $T$ akzeptierte Sprache, falls Anfragen gemäß $A$ beantwortet werden.

\end{definition}




\begin{beispiel}
    
    Sei $STCONN = \{ (G, s, t) \ |\  \exists \text{ Pfad von } s \text{ nach } t \ \ \in G \} $

    Feststellen, ob ein Graph $G$ einen nichttrivialen Zyklus enthält. Zähle alle Paare $(u,v)$ auf und frage jeweils $(G, u, v)$ und $(G, v, u)$ ab.

    Damit ist eine Maschine $T$ beschrieben, sodass $CYCLE = L^{STCONN}(T)$.\\
    Die Laufzeit von $T$ ist $|G|^2$ (insbesondere polynomiell). Es ist also $CYCLE \in P^{STCONN}$.\\
    Nachdem nun $STCONN \in \ComplexityClassP$ folgt $CYCLE \in \ComplexityClassP$.

\end{beispiel}


\begin{definition}

    Sei $A \subseteq \Sigma^\ast$.

    Man definiert $\ComplexityClassP^A$ als die Menge aller Sprachen $L$ sodass eine deterministische Turingmaschine $T$ existiert mit $L = L^A(T)$ und $DTIME(T^A(x)) \leq p(|x|)$ für eine Polynom $p$.

    Analog $\ComplexityClassNP^A$.
    
\end{definition}

\textbf{Beobachtung:}
$A \in \ComplexityClassP \Rightarrow \ComplexityClassP^A = \ComplexityClassP \land \ComplexityClassNP^A = \ComplexityClassNP$.

    
\begin{beispiel}
    
    $SAT \in \ComplexityClassP^{SAT}$

    $NODE-COVER \in \ComplexityClassP^{SAT}$

    $\ComplexityClassNP \in \ComplexityClassP^{SAT}$


    $ TAUT = \{ \phi \ |\  \phi \text{ allgemeingültig} \} \in \ComplexityClassP^{\ComplexityClassNP}$


    $ IMPL = \{ (\phi, \psi) \ |\  \phi \text{ allgemeingültig} \Rightarrow \psi \text{ allgemeingültig} \} \in \ComplexityClassP^{\ComplexityClassNP} $


    $ CIRCUIT-MIN = \{  (\text{Schaltkreis } C,  A) \ | \ 
            \text{Schaltkreis } C' \text{ der Größe } \leq k \text{ und } C \equiv C' \}  \in \ComplexityClassNP^{SAT}$

\end{beispiel}



Wr zeigen jetzt die folgenden zwei Sätze, aufgrund derer kein Beweis für $\ComplexityClassP = \ComplexityClassNP$ oder $\ComplexityClassP \neq \ComplexityClassNP$ existieren kann, welcher in Gegenwart von Orakeln auch funktioniert:

\begin{satz}
    
    $$ \exists A \in \Sigma^\ast: \ComplexityClassP^A = \ComplexityClassNP^A $$

\end{satz}

\begin{beweis}
    
    $A = \{ ( T, x, 0^k )  \ |\  
        \text{ deterministische Turingmaschine } T \text{ akzeptiert } x \text{ und benutt dabei } \leq k \text{ Bandzellen}  \}
    $

    $A$ ist offensichtlich entscheidbar.

    Sei $T$ eine nichtdeterministische Orakel-Turingmaschine und $p(n)$ ein Polynom, das die Laufzeit von $T$ beschränkt und somit auch die Größe aller Orakelanfragen.\\
    Wir müssen eine deterministische polynomiell zeitbeschränkte Turingmaschine $T'$ mit Orakel $A$ bauen, sodass $L^A(T') = L^A(T)$.

    Zunächst konstruieren wir eine deterministische Turingmaschine $T_{HILF}$, die ohne Orakelbenutzung die Sprache $L^A(T)$ entscheidet, indem alle Orakelanfragen ``mit Bordmitteln'' (also selbst) beantwortet werden unter Verwendung der Entscheidbarkeit von $A$.

    Vollständige Berechnungssequenzen einer Berechnung, deren Bandplatz $\leq k$ ist und $t$ Schritte lang ist, benötigen Platz $\bigO(k*t)$.\\
    Orakelanfragen einer Berechnung von $T$ auf $x$ (Eingabe) haben die Form $(S, y, 0^k)$ wobei $|S|, |y|, k \leq p(|x|)$.\\
    Wir verwenden also jetzt 3 Hilfsbänder, eines für die nichtdeterministische Berechnung von $T$ auf $x$, die wir der Reihe nach alle simulieren, eines für Orakelanfragen, eines für die Beantwortung der Orakelanfragen.\\
    Diese Maschine ist deterministisch und benötigt auf ihren Bändern höchstens $q(|x|)$ Platz, wobei $q$ ein von $p$ abgeleitetes Polynom ist (in etwa $q(n) = \bigO(p(n)^2$).\\

    Die eigentlichte Maschine $T'$ arbeitet jetzt wie folgt:\\
    Gegeben Eingabe $x$, schreibe $(T_{HILF}, x, O^{q(|x|)})$ auf das Orakelband. Falls $q_{yes}$, dann akzeptiere. Falls $q_{nein}$, dann verwerfe.

\end{beweis}


\begin{satz}
    
    $$ \exists B \in \Sigma^\ast: \ComplexityClassP^B \neq \ComplexityClassNP^B $$

\end{satz}

\begin{beweis}
    
    Falls $B \subseteq \Sigma^\ast$, 
    definiere $L_B = \{  0^k \ |\   \exists x \in \Sigma^\ast: |x| = k \land x \in B  \} $ \\
    Offensichtlich ist $L_B \in \ComplexityClassNP^B$, egal was $B$ ist.

    Es gilt jetzt, $B$ so zu wählen, dass für jede polynomiell zeitbeschränkte deterministische Orakel-Turingmaschine gitl: $L_B \neq L^B(T)$.

    Sei $i \mapsto T_i$ eine effektive Aufzählung von Orakel-Turingmaschinen, sodass $DTIME(T_i^x(x)) \leq |x|^i + i$ (alternativ $(|x| + 2)^i$ wie letztes Mal).\\
    Für alle deterministischen Orakel-Turingmaschinen $S$ und alle Orakel $x$ muss $i$ existieren, sodass $L(S^x) = L(T_i^x)$. $T_i$ ist die durch $i$ beschriebene Orakel-Turingmaschine künstlich auf Laufzeit $n^i + i$ beschränkt. Jetzt muss also für jedes $i$ ein $n_i$ existieren, sodass $T_i^B(0^{n_i})$ akzeptiert und $B$ enthält kein Wort der Länge $n_i$ (dann ist nämlich $0^{n_i} \notin L_B$), oder aber $T_i^B(0^{n_i})$ verwirft und $B$ enthält ein Wort $x_i$ mit $|x_i| = n_i$, denn dann ist $0^{n_i} \in L_B$.\\
    Dann ist in der Tat $L_B \notin \ComplexityClassP^B$.

    \textbf{Beobachtung:}
    Wenn $T_i^X(x)$ nach $t$ Schritten hält und $U$ aus Wörtern $y$ mit $|y| > t$ besteht, dann gilt 
    $T_i^{X \cup U}(x) \text{ akzeptiert }  \Leftrightarrow  \ T_i^X(x) \text{ akzeptiert}$.

    % TODO beweis fertig machen
    
\end{beweis}




% chapter np_und_p (end)

%!TEX root = 0-main.tex

% Author: Philipp Moers <soziflip@gmail.com> 



\datum{26.11.15}

\chapter{Platzkomplexität} % (fold)
\label{cha:platzkomplexitaet}
    



\section{Platzkonstruierbare Funktionen}


\begin{definition}
    
    Eine Funktion $s(n)$ heißt \definiere{platzkonstruierbar} genau dann, wenn eine deterministische Turingmaschine existiert, die bei Eingabe $0^n$ genau $s(n)$ Bandfelder beschreibt und dann hält.

\end{definition}

\begin{beispiel}

    Alle Polynome mit Koeffizienten $\in \Q^+$, die Wurzelfunktion, die Logarithmus-Funktion, die Potzenfunktion usw. sind platzkonstruierbar.
    
\end{beispiel}




\section{Platzverbrauch einer Turingmaschine}

\begin{definition}

    Der \definiere{Platzverbrauch einer Turingmaschine} (deterministisch oder nichtdeterministisch) bei Eingabe $x$ ist 

    \begin{itemize}

        \item \textbf{erste Definition}\\ 
        die Größe des beschriebenen Teils aller Bänder am Ende der Berechnung. (Mit dieser Definition ist der Platzverbrauch stets $\geq |x|$).

        \item \textbf{zweite Definition}\\ 
        die Endgröße aller anderen Bänder, wobei das Eingabeband nicht überschrieben werden darf.

    \end{itemize}

    Die zweite Definition ist Standard,w enn sublineare Platzschranken betrachtet werden, zum Beispiel $\log(n)$. Oberhalb von $\bigO(n)$ sind die beiden Definitionen äquivalent.
    


    Notation: $DSPACE_M (x)$ und $NSPACE_M (x)$

    $DSPACE_M (s(n)) = \{   L  \ |\   \exists DTM M : L = L(M) \land DSPACE_M(x) = \bigO(s(|x|)) \}$

    $NSPACE_M (s(n)) = \{   L  \ |\   \exists DTM M : L = L(M) \land NSPACE_M(x) = \bigO(s(|x|)) \}$

    $PSPACE = \bigcup_{k \geq 0} DSPACE(n^k) $ (polynomieller Platz)

    $LINSPACE = DSPACE(n)$

    $LOGSPACE = DSPACE(\log n)$ (auch als $L$ bezeichnet)

    $NLOGSPACE = NSPACE(\log n)$ (auch als $NL$ bezeichnet)


\end{definition}


\begin{beispiel}
    
    STCONN (Erreichbarkeit in gerichteten Graphen) ist $\in NLOGSPACE$ (rate Pfad) und $\in LINSPACE$ (Tiefensuche/Breitensuche)

\end{beispiel}


Es gibt eine triviale, aber wissenswerte Beziehung zwischen Zeit- und Platzkomplexität:

% $$ NSPACE(s(n)) \subseteq DTIME (2^{\bigO(s(n))}) $$      % not sure
$$ DSPACE(s(n)) \subseteq DTIME (2^{\bigO(s(n))}) $$

% Alle Berechnungsfolgen können aufgezählt werden.
Hat die Berechnung nach $2^{c*s(n)}$ Schritten nicht geendet, so kann abgebrochen werden wegen Wiederholung einer globalen Konfiguration. (Tatsächlicher Platzverbrauch $\leq c * s(n)$)



$$ DTIME(t(n)) \subseteq DSPACE (t(n)) $$

Mehr Platz als Laufzeit kann nicht angefordert werden.





\begin{satz}

    Für deterministische Einband-Turingmaschinen $T$ gilt:

    $DTIME_T(x) = \bigO(t(|x|) \Longrightarrow  L(T) \in DSPACE(\sqrt{t(n)})$

    
\end{satz}

Für Mehrband-Turingmaschinen gibt es einen ähnliches Satz, bei dem der Platz allerdings etwas größer ist. Dass er für Einband-Turingmaschinen so gut ist, ist gewissermaßen kurios.





\section{Platzhierarchiesatz}

\begin{satz}
    
    ``Echt mehr Platz hilft auch mehr.''

    \textit{Für genaue Aussage und Beweis siehe z.B. Papadimitrion}
\end{satz}

Wichtige Konsequenz:

$$ LOGSPACE \subset PSPACE $$

$$ LOGSPACE \subseteq NLOGSPACE \subseteq \ComplexityClassP \subseteq \ComplexityClassNP \subseteq PH \subseteq PSPACE $$

Von jeder dieser Inklusionen ist unbekannt, ob sie echt sind. Mindestens eine muss aber echt sein.






\section{Zusammenhänge von Platzkomplexitätsklassen}


\begin{satz}
    
    \definiere{Satz von Savitch}

    Für eine platzkonstruierbare Funktion $s(n) \geq \log(n)$ ist \\
    $ NSPACE(s(n)) \subseteq DSPACE(s(n)^2)$

    (Vergleiche
    $ NTIME(t(n)) \subseteq DTIME(2^{\bigO(t(n))})$
    )

\end{satz}

\begin{beweis}
    
    Sei eine nichtdeterministische Turingmaschine $T$ gegeben mit Platzbedarf $S = c * s(|x|)$ bei Eingabe $x$. Wir betrachten eine Kodierung der globalen Konfigurationen von $T(x)$ durch Wörter der Länge $S$ und o.B.d.A gebe es exakt eine akzeptierende Endkonfiguration $s_{ACC}$. (Alle Bänder am Ende löschen, d.h. mit $0$ überschreiben.)

    $
    x \in L(T) 
        \Longleftrightarrow 
    s_{ACC} \text{ von } s_{INI} \text{ aus in } \leq 2^S \text{ Schritten erreichbar}
    $

    Hier steht $s_{INI}$ für die Startkonfiguration bei Eingabe $x$. $2^S$ ist die Gesamtzahl der Konfigurationen.

    Das heißt $s_{ACC}$ ist von $s_{INI}$ aus im Graphen der Konfigurationen erreichbar (Spezialfall von STCONN).

\end{beweis}


\textbf{Notation:}

$s \rightarrow_T s'$: $s'$ ist 1-Schritt-Folgekonfiguration von $s$ in $T$ und kann in $LOGSPACE$ entschieden werden.

$REACH(s, s')$: $s \rightarrow^\ast s'$ ($s'$ ist von $s$ erreichbar)

$REACH(s, s', i)$: $s \rightarrow^{\leq 2^i} s'$ ($s'$ ist von $s$ in weniger als $2^i$ Schritten erreichbar)


$ x \in L(T) \Longleftrightarrow REACH(s_{INI}, s_{ACC}, S)$

Es gilt 
$$REACH(s, s', 0) \Longleftrightarrow s = s' \lor s \rightarrow_T s'$$
\hspace{4cm}($2^0 = 1$)
$$REACH(s, s', i+1) \Longleftrightarrow \exists \check{s} : REACH(s, \check{s}, i) \land REACH(\check{s}, s', i)$$ 
\hspace{4cm}($2^{i+1} = 2 * 2^i$)\\
Dies liefert eine rekursive Implementierung von $REACH(s,s',i)$ \\
\hspace{4cm}($\exists \check{s} \leadsto \text { for } \check{s} \in \text{ globale Konfigurationen}$)


Der Rekursionsstack hat Tiefe $S$ (Toplevel-Aufruf $REACH(s_{INI}, s_{ACC}, S)$).
\\
Jeder Activationrecord hat Größe $\bigO(S)$ genauer gesagt $2 S$ für die beiden Parameter $s, s', \log(S)$ für die Parameter $i$. Wenn gewünscht noch ein weiteres $S$ für die for-Schleife.


Die Gesamtgröße des Stacks ist beschränkt durch $S * \bigO(S) = \bigO(S^2)$.


Historisch wurde zunächst gezeigt, dass STCONN in $DSPACE(\log(n)^2)$ liegt. Der Satz von Savitch kann auch hieraus abgeleitet werden.









\datum{30.11.15}




\begin{satz}

    \definiere{Satz von Immerman-Szelepcsényi}

    Sei $s(n) \geq \log(n)$.

    Dann ist $NSPACE(s(n)) = co\text{--}NSPACE(s(n))$

    Wichtiger Spezialfall: $co\text{--}STCONN \in NSPACE(\log(n)) = NL = NLOGSPACE$
    \\
    Die allgemeine Behauptung kann aus diesem Spezialfall leicht gefolgert werden (durch den Graph der globalen Konfiguration).


\end{satz}

\begin{beweis}

    Es sei eine nichtdeterministische Turingmaschine $T$ vorgelegt und $NSPACE_T(x) \leq c * s(|x|)$. Wir müssen eine nichtdeterministische Turingmaschine $T'$ konstruieren, sodass $NSPACE_{T'}(x) = \bigO(s(|x|))$ und $x \in L(T') \Leftrightarrow x \notin L(T)$.
    \\
    Es existiert akzeptierende Berechnung von $T'$ auf x genau dann, wenn alle Berechnungen von $T$ auf $x$ verwerfen.

    Sei $x$ fixiert und o.B.d.A $\Sigma = \{0,1\}$.
    \\
    Schreibe $S_{INI}$ für die globale Startkonfiguration von $T$ auf $x$
    und $S_{ACC}$ für die (o.B.d.A. einzige) akzeptierende globale Konfiguration.
    Weiter sei $S = c' *  s(|x|)$ so gewählt, dass alle globalen Konfigurationen durch $0/1$-Strings der Länge $S$ kodiert werden.
    \\
    $s \rightarrow_T s'$ bedeute, dass $s'$ in einem Schritt aus $s$ hervorgehen kann (Das kann in Platz $\log(S)$ entschieden werden).
    \\
    $T'$ soll nun $x$ akzeptieren genau dann, wenn kein Pfad (der Länge $2^S$) von $s_{INI}$ zu $s_{ACC}$ existiert.
    (Anzahl der globalen Konfigurationen ist kleiner als $2^S$. Ein einziger Pfad braucht, wenn er voll ausgeschrieben wird, schon Platz $S * 2^S \notin \bigO(s(|x|))$.)

    \textbf{Vorbemerkung:}\\
    Nehmen wir an, dass die Anzahl $N$ der von $s_{INI}$ aus erreichbaren globalen Konfigurationen bekannt ist bzw. berechnet werden kann.

    Wir zählen der Reihe nach alle globalen Konfigurationen auf (geht mit Platz $\bigO(S)$) und raten für jede von denen einen Pfad von $S_{INI}$ dorthin. Durch Mitführen eines Zählers haben wir am Ende der Aufzählung die Anzahl derjenigen Knoten, für die das gelungen ist.

    \begin{codebox}[javascript]
function A(...) {
    cnt = 0;
    for (s in globalConfigs) {
        pfad = guessPath();
        if (pfad.endsAt(s))
            cnt++;
        if (s = s_acc)
            return "reject";
    }
    if (cnt == N)
        return "accept";
    else
        return "reject";
}

function guessPath() {
    s = s_ini;
    for (i = 1; i <= 2^S; i++) {
        sX = guessNonDet({0,1}^S);
        if (s2 -> sX) {
            s2 = sX;
        } else {
            return "reject";
        }
        b = guessNonDet({0,1});
        if (b) {
            break;
        }
    }
}
    \end{codebox}

    Falls $N$ die Anzahl der von $s_{INI}$ aus erreichbaren globalen Konfigurationen ist, so \underline{kann} A akzeptieren genau dann, wenn $s_{ACC}$ von $s_{INI}$ unerreichbar ist. 
    \\
    Begründung ``$\Rightarrow$'': Falls A akzeptiert, dann ist $s_{ACC}$ tatsächlich unerreichbar, weil alle $N$ erreichbaren Konfigurationen in der for-Schleife als solche erkannt wurden und $s_{ACC}$ nicht unter ihnen war.
    \\
    Begründung ``$\Leftarrow$'': Falls $s_{ACC}$ unerreichbar ist, so kann A akzeptieren, indem bei jedem der von $s_{INI}$ aus erreichbaren s tatsächlich ein entsprechender Pfad geraten wird.


    Grobe Struktur des Algorithmus für $T'$:
    \begin{codebox}[javascript]
N = 1;
for (i = 1 ... 2^S) {
    // Invariante: Anzahl der von s_ini aus in weniger als i-1 Schritten erreichbaren Konfigurationen ist gleich N
    updateN();
}
A();
    \end{codebox}
    Der Block \codeline{updateN()} wird selbst Nichtdeterminismus enthalten, in dem Sinne, dass die gesamte Berechnung verwerfend abgebrochen werden kann. Passiert das nicht, dann ist $N$ korrekt aktualisiert und bei passender Wahl der nichtdeterministischen Entscheidungen, passiert das auch.

    \begin{codebox}[javascript]
function updateN() {
    cnt = 0;
    for (s in globalConfigs) {
        reachable = false;
        cnt2 = 0;
        // alle N Stück die von s_ini aus in weniger als i-1 Schritten erreichbar sind, aufzählen
        for (sCheck in globalConfigs) {
            pfad = guessPath(); 
            if (pfad.endsAt(sCheck)) {
                cnt2++;
                if (sCheck == s || sCheck -> s)
                    reachable = true;
            }
        }
        if (cnt2 != N)
            return "reject";
        else if (reachable)
            cnt++;
    }
    N = cnt;
}
    \end{codebox}

    Am Ende von \codeline{updateN} hat entweder N den korrekten Wert oder es wurde verworfen.
    Es ist möglich, die nichtdeterministischen Entscheidungen so zu treffen, dass der korrekte Wert geliefert wird, sodass nicht verworfen wird.


\end{beweis}






% chapter platzkomplexitaet (end)

% \input{5-alternierung-und-hierarchien.tex}
% \input{6-schaltkreise.tex}
% %!TEX root = 0-main.tex

% Author: Philipp Moers <soziflip@gmail.com>



\datum{07.01.16}


\chapter{Probabilistische Algorithmen} % (fold)
\label{cha:probabilistische_algorithmen}



\section{Miller-Rabin Primzahltest Einführung}

Gegeben: $n \in \N$ binär kodiert, $\text{Länge}(n) = \log(n)$

Gefragt: Ist $n$ prim?

Man müsste testen bis $\sqrt{n}$
Anzahl der Zahlen $< \sqrt{n}$: $\sqrt{2^{\log(n)}} = 2^{\frac{1}{2} \log(n)}$, d.\,h. exponentiell in $\text{Länge}(n)$.
Das ist nicht effizient, insbesondere $\notin \ComplexityClassP$.
% todo wo ist das P


\begin{lemma}

    Sei $G$ abelsche Gruppe (endlich) und $U$ eine Untergruppe von $G$.

    Dann ist $|U|$ ein Teiler von $|G|$.

\end{lemma}

\begin{korollar}

    Falls $ggt(x, n)$ = 1, d.\,h. $x \in \Phi(n)$, dann $x^{\phi(n)} = 1 \mod n$,
    wobei $\phi(n) = |\Phi(n)|$ die Anzahl der zu $n$ teilerfremden Zahlen kleiner $n$ ist.

\end{korollar}

\begin{korollar}

    Falls $n$ prim ist, so gilt $x^{n-1} = 1 \mod n$

\end{korollar}


\begin{lemma}

    Falls $x \notin \Phi(n)$, so ist $x^{n-1} \neq 1 \mod n$.

\end{lemma}
\begin{beweis}

    Sei $b = ggt(x, n) > 1$, $v = x^{n-2} \mod n$.

    $x^{n-1} = v * x + u * n $ für ein $u \in \Z$

    $b | RHS \Rightarrow x^{n-1} \neq 1$

\end{beweis}



\section{Fermat-Primzahltest}

Gegeben: $n \in \N$

Wähle zufällig $x \in \{1 \dots n-1\}$.
\\
Bilde $a := x^{n-1} \mod n$ mittels ``Repeated Squaring''.
\\
Falls $a = 1$, so antworte ``ist prim''.
\\
Falls $a \neq 1$, so antworte ``ist zusammengesetzt''.

Ist $n$ tatsächlich prim, so wird mit Sicherheit geantwortet ``ist prim'' (siehe Korollar).
\\
Ist $n$ zusammengesetzt und $x \notin \Phi(n)$, dann wird korrekt geantwortet ``ist zusammengesetzt''.
\\
Wenn $x \in \Phi(n)$, so kann trotzdem $a = 1$ sein, also korrekt geantwortet werden.

Falls $U \neq \Phi(n)$, dann ist $|U|$ ein echter Teiler von $phi(n)$, also $\leq \frac{1}{2} n$.
D.\,h. für mehr als 50\% der $x$ ist dann $a \neq 1$.

Falls aber für alle $x \in \Phi(n)$ gilt $x^{n-1} = 1 \mod n$ und doch $n$ zusammengesetzt ist, dann funktioniert der Fermat-Test nicht gut. Solche $n$ heißen Carmichaelzahl, falls $n$ nicht prim, aber $x^{n-1} = 1 \mod n \forall x \in \Phi(n)$. Solche Carmichaelzahlen sind sehr selten.



\begin{satz}

    Ist $n$ Carmichaelzahl, so ist der Anteil der $x \in \{1 \dots n-1\}$, sodass $\exists i: ggt(x^{\frac{n-1}{2^i}} - 1, n) > 1$ größer oder gleich 75\%.

\end{satz}



\section{Miller-Rabin Primzahltest}

Gegeben $n \in \N$

Wähle zufällig $x \in \{1 \dots n-1\}$.
\\
Bilde $a = x^{n-1} \mod n$.
\\
Falls $a \neq 1$, so ist $n$ sicher zusammengesetzt.
\\
Falls $a = 1$, berechne $ggt(x^{\frac{n-1}{2^i}} - 1, n)$ für $i = 1 \dots \log(n)$. Falls einmal $> 1$, so ist $n$ sicher zusammengesetzt. Sonst antworte ``ist prim''.

Fehlerwahrscheinlichkeit $\leq 50\%$.





\section{Testen von polynomiellen Identitäten}

% TODO vbl?
Gegeben: zwei Polynome $p(x), q(x)$ mit Koeffizienten aus $\Z$ und $n$ Vbl $x = x_1 \dots x_n$.

Gefragt: Ist $p(x) = q(x)$?


\begin{beispiel}

    $p = (a^2 + b^2 + c^2 + d^2)(A^2 + B^2 + C^2 + D^2)$

    $q = (aA + bB + cC + dD)^2 + \\
         (aB + bA + cD + dC)^2 + \\
         (aC + bD + cA + dB)^2 + \\
         (aD + bC + cB + dA)^2 \\
    $

    Hier ist in der Tat $p = q$.

\end{beispiel}

Für diese Aufgabe ist kein Polynomialzeitverfahren bekannt.
Ausmultiplizieren hat im Allgemeinen exponentielle Laufzeit, falls $*$ und $+$ geschachtelt auftreten.



\textbf{Probabilistisches Verfahren:}

Wähle zufällig $x \in \{-nd, \dots 0, 1, 2, \dots , nd\}^n$,
wobei $n = \text{ Anzahl Vbl } $ und $  d \geq Grad(p), Grad(q)$.

Wir zeigen jetzt, dass die Fehlerwahrscheinlichkeit kleiner oder gleich 50\% ist.

\begin{lemma}

    \definiere{Lemma von Schwarz-Zippe}

    Sei $p$ ein Polynom vom Grad $d$ in $n$ Variablen und $p \neq q$.

    Dann hat $p$ im Bereich $\{-N, \dots N\}^n$ höchstens $dn(2N+1)^{n-1}$ Nullstellen.

\end{lemma}

Falls $N = nd$, dann ist $\frac{(2N+1)^{n-1} * dn}{(2N+1)^n}  =  \frac{dn}{2N+1} = \frac{dn}{2dn+1} \leq 50\%$.








\datum{11.01.16}


\begin{definition}

    Ein \definiere{Perfektes Matching} zu Graph $G= (V, E)$ ist eine
    Kantenauswahl $T \subseteq E$, sodass
    $u, v \in T, u \neq v \rightarrow u \cap v = \emptyset$.
    (Die ausgewählten Kanten haben keine Punkte gemeinsam.)

    Ein Graph ist perfekt, wenn $\bigcup T = V$.
    (Alle Knoten werden gematcht.)

\end{definition}


\begin{satz}
    Sei $G=(V,E)$ ein Graph und $A$ die zugehörige Adjazenzmatrix.
    $G$ hat ein perfektes Matching genau dann, wenn $det(A) \neq 0$.
\end{satz}



\begin{definition}
    Eine \definiere{Probabilistische Turingmaschine} ist wie eine nichtdeterministische
    Turingmaschine aufgebaut. Zu jedem Paar $q, a$ gibt es exakt zwei Quintupel
    (ggf.\ künstlich aufblähen).
\end{definition}

Deutung:
Passendes Quintupel wird jeweils mit Wahrscheinlichkeit 50\% ausgewählt.
Die Wahrscheinlichkeit, dass $T$ die Eingabe $x$ akzeptiert, ist
$\frac{\text{Anzahl akzeptierender Berechnungen}}{\text{Anzahl aller Berechnungen}}$.

Polynomialzeitbeschränkt, falls Laufzeit $\leq p(|x|)$ unabhängig von den Zufallsentscheidungen.


\begin{definition}
    Eine Sprache $X \subseteq \Sigma^\ast$ ist in der Komplexitätsklasse $\ComplexityClassPP$,
    wenn eine Probabilistische Turingmaschine $T$ existiert, die
    \begin{itemize}
        \item polynomiell zeitbeschränkt ist,
        \item $x \in X \Leftrightarrow \text{Akzeptanzwahrscheinlichkeit} \geq 50\%$.
    \end{itemize}
\end{definition}

\begin{satz}
    $\ComplexityClassP \subseteq PSPACE$
\end{satz}
\begin{beweis}
    Alle Berechnungen durchprobieren (Platz jeweils wiederverwenden) und Anzahl der
    akzeptierenden Berechnungen mitzählen.
\end{beweis}

\begin{satz}
    $\ComplexityClassNP \subseteq PSPACE$
\end{satz}
\begin{beweis}
    Sei eine nichtdeterministische Turingmaschine $T$ für $X$ gegeben
    (O.B.d.A. im 2. Nachfolgequintupelformat).
    Zunächst mit 50\% Wahrscheinlichkeit einfach so akzeptieren,
    anschließend $T$ als Probabilistische Turingmaschine fahren.
    \\
    $x \in X$: Wahrscheinlichkeit der Akzeptanz $\geq 50\%$
    \\
    $x \notin X$: Wahrscheinlichkeit der Akzeptanz $= 50\%$
\end{beweis}

\begin{satz}
    (Beispiel Spielman et al.)

    $\ComplexityClassPP$ ist unter $\cap, \cup$ abgeschlossen.
\end{satz}
Der Beweis dazu ist kompliziert.



\begin{definition}
    Komplexitätsklassen $\ComplexityClassR, \ComplexityClassRP$

    $X \in \ComplexityClassR$ genau dann, wenn eine polynomiell zeitbeschränkte Probabilistische Turingmaschine $T$ existiert mit
    \begin{itemize}
        \item $x \in X \Rightarrow$ Wahrscheinlichkeit der Akzeptanz $\geq 50\%$
        \item $x \notin X \Rightarrow$ Wahrscheinlichkeit der Akzeptanz $= 0$
    \end{itemize}
\end{definition}

\begin{beispiel}
    $PRIM \in \ComplexityClassR$
\end{beispiel}


\begin{definition}
    Komplexitätsklasse $\ComplexityClassBPP$

    $X \in \ComplexityClassBPP$ genau dann, wenn eine polynomiell zeitbeschränkte Probabilistische Turingmaschine $T$ existiert mit
    \begin{itemize}
        %\item $x \in X $: Wahrscheinlichkeit der Akzeptanz $\geq 75\%$
        \item $x \in X $: Wahrscheinlichkeit der Akzeptanz $> 75\%$
        \item $x \notin X $: Wahrscheinlichkeit der Akzeptanz $\leq 25\%$
    \end{itemize}
\end{definition}

\begin{satz}
    Sei $A \in \ComplexityClassBPP$ und $q(n)$ ein Polynom.
    \\
    Es existiert eine Probabilistische Turingmaschine $T$ für $A$
    mit Fehlerwahrscheinlichkeit $\leq e^{-q(|x|)}$ für Eingabe $x$.
\end{satz}
\begin{beweis}
    Beweisidee: $A$ auf Eingabe $x$ wiederholt, d.\,h. $t = t(|x|)$ mal ablaufen lassen.
    Am Ende diejenige Antwort, die häufiger gegeben wurde, nach außen reichen.

    $x \in A$ Anzahl der Akzeptanzen binomialverteilt mit Parameter $p \geq 75\%$.
    Sei $S$ diese Anzahl (Zufallsvariable).
    Wir möchten $Pr(S \leq \frac{t}{2}) \leq e^{-q(|x|)}$.

    Es gilt allgemein die Chernoff-Schranke.
    (\textit{siehe Details im Buch})

\end{beweis}











\datum{14.01.2016}

\section{Chernoff Schranke}
Seien $X_1 \dots X_n$ $0-1$
Zufallsvariablen mit $Pr(X_i=1) = P_i$

Setze $S=X_1 + \dots + X_n$ \\
$\mu = E[S] = p_1 \dots  + p_n$

Für jedes $S > 0$

$PR (S>(1+S)) \mu \leq (\frac{e^\delta}{(1+\delta)^{+\delta}})^\delta$
Falls zusätzlich $\delta < 1$ \\
$Pr(S\leq (1- \delta) \mu ) \leq e^{\frac{- \delta^2}{2} * \mu}$

In der Literatur gibt es andere Versionen.

\paragraph{Anmerkung}
t-fache unabhängige Ausführung eines BRP-Algorithmus für
$A\subseteq \Sigma^*$ bei Eingabe x. $x \in A$: $\mu = 0,75 * t$

Fehlerwahrscheinlichkeit:
$Pr(S\leq \frac{t}{2}) = Pr(S \leq (1 - \delta) \mu)$
Für $\delta = \frac{2}{3}$

t ist also so zu wählen, dass

$e ^ {\frac{\frac{-2}{3}^2}{2} * \frac{3}{4} * t} \leq e^{-q(|x|)}$

Also $q(|x|) \geq \frac{\frac{3}{2}^2}{2} * \frac{3}{x} * t$

$q(|x|) \geq 24 t$

$t \notin A : \mu = \frac{1}{4} * t $

$ PR( S\geq \frac{t}{s} = PR (S \grq (1+1) \mu ) $
$ \geq \frac{e}{4}^\frac{t}{4} = (e^{-0.386})^\frac{t}{4}$

Es muss also
$0.386 * \frac{t}{4} \leq q(|x|)$
Wahr, wenn $t(|x|)=24 * g(|x|)$

\subsection{Anwendung}
Satz: $BPP \subseteq \frac{P}{Poly}$
$\frac{P}{Poly} = \{A | \forall n \exists$  Schaltkreis $C. |C| \leq p(n) \forall x . |x| < n : x \in A \Leftrightarrow C(x) = 1\} $

$SAT \in \frac{P}{Poly} \Rightarrow PH$ kollabiert

\begin{beweis}
Sei ein BPP Algirithmus M für A gegeben. OBdA(vorhergehender Satz) können wir annehmen, dass

$ x \in A  \Rightarrow Pr (M akzeptiert . x) \geq 1-2^{-(|x| + 1)}$

$ x \notin  A  \Rightarrow Pr (M akzeptiert . x) \leq 1-2^{-(|x| + 1)}$

Wir nehmen außerdem an, dass $| \Sigma  | = 2$ (sonst ersetze 2 durch $| \Sigma|$).

Anteil der Zufallsentscheidungen, der für ein bestimmtes $x\in \Sigma^*$
ungünstig ist, ist $\leq 2 ^{-(|x| + 1)}$, d.h. $\leq 2^{-(n+1)}$ falls
$|x| = n$

Der Anteil der Zufallsentscheidungen, der ffür mindestens ein $x\in \Sigma^*$
ungünstig ist, also $\leq 2^n * 2^{-(n+1)} \leq \frac{1}{2}$

Der Anteil der Zufallsentscheidungen die für jedes x mit $|x|$ = n günstig  sind,
ist $>0$, ja sogar $> \frac{1}{2}$. Der Schaltkreis $C$, der für alle Inputs der
Größe $n$ funktioniert ergibt sich aus $M$ in dem eine dieser universell
günstigen Entscheidungen festverdrahtet wird.
\end{beweis}

BPP, RP, co-RP \checked

ZPP : Antwort immer richtig mit $W\leq 50$ darf "dont know'' geantwortet werden. Laufzeit in jedem Fall polynomiell.

Offensichtlich $ZPP = RP \cap co - RP$

Las Vegas Algorithmus (verallgemeinert ZPP) Liefert nur richtige Ergebnisse aber manchmal($W<50\%$) kein Ergebnis ("dont know'')

Monte Carlo Algorthmus Liefert Ergenbis, dass mit $W \leq 25\% falsch ist (verallgemeinert BPP)$

\begin{beispiel}
für Las Vegas

Randomisiertes Quicksort (Laufzeitschranke $n log n$ mit laufen lassen
"dont know'' falls nicht richtig)
\end{beispiel}

Allg. ist ein Las Vegas Alg. ersetzbar durch einen, der immer richtig
antwortet, aber mit einer gewissen Wahrscheinlichkeit nicht
terminiert. Letzteres mit Wahrscheinlichkeit $0$.

(Wiederholung)
Folgene formale Version:
$ZPP = \{A |$ es existiert ein Entscheidungsverfahren, für A dessen Laufzeit eine Zufallsvariable ist mit polynomiellem Erwartungswert $\}$. beschränkt.

\section{Interaktive Beweissysteme}
\begin{itemize}
\item Polynomiell laufzeitbeschränkter Verifizierer (``wir''): $V$
\item Mit umbeschränkter Rechenkraft ausgestatteter Beweiser (``sie''): $P$
\end{itemize}

% Jetzt kommt eine GRafik
%Eingabeband

$P$ und $V$ haben spezielle Zustände zur Beendigung von Phasen.

Bei gegebnere Eingabe $x$ auf dem Eingabend finden eine bestimmte Zahl von Runden statt.
$Zahl \leq poly (|x|)$
Zu Beginn jeder Runde schreitet $V$ ein Wort aufs Frageband. Anschließend schreibt $P$ eine Antwort aufs Antwortband.

Zum Schluss(alle Runden vorbei) entscheidet ob akzeptiert oder nicht.

\begin{definition}
Sei $X \subseteq \Sigma^*$. Eine polynomiell beschränkte Turingmaschine $V$ ist polynomiell verifizierbar für $x$
genau dann wenn $ \exists$ Prover $P$
$\forall x \in A$ . $V \leftrightarrow P$ auf $x$ akzeptiert. $\forall x \notin A$ . $V \leftrightarrow P$   akzeptiert auf $x$ nicht.

Falls Verifizierer deterministische polynomiell beschränkt
Turingmaschine sind, so spricht man von der Klasse DIP (falls Problem für
die ein deterministischer polynomiellialzeit Verifizierer existiert.)
\end{definition}

\begin{satz}
DIP = NP

\begin{description}
\item["$\subseteq$''] Verwendet Formulierung von NP mit Polynom großen "Lösungen''.
\item["$\supseteq$''] Sei $V$ ein DIP Verifizierer für $X$ und P der
  zugehörigem Prover. Komplettes Transcript raten bzw. als Lösung
  verwenden.
\end{description}
\end{satz}

\begin{definition}
$ \subseteq \Sigma^*$ ist in $TP$
falls eine probailistischer polynomialzeit beschrkänkter Verifizierer $V$
existiert so, dass $\exists$ Prover $P$

$ x\in A : Pr(V \leftrightarrow P$ auf $x$ akzeptiert) $\geq 75\%$.

$ x \notin A : \forall$ Prover $P$.

$P_1 (V \leftrightarrow P$ auf $x$ akzeptiert) $\leq 25\%$.

Evidenz für IP ``$\supsetneqq$'' $NP$.

Sei $G$ das Problem Graph-Isomorphismus. Offensichtligh $GI \in NP$ Isomorphismus vorlgegen.

Man weiß nicht ob $\overline{GI} \in NP$.

Aber: $\overline{GI} \in IP$.

$V:$

\begin{itemize}
\item Gegenben: $G$,  $H$ Graphen
\item Wähle zufällig (1:1) $b\in \{0,1\}$
\item if b then K:= G else K:=H
\item Wähle zufällige Permutation $\pi$ auf den Knoten von $K$ und stelle an $P$ die Fragen $\pi(K)$ $P$ antwortet mit $b\in \{0,1\}$
Akzeptiere, falls $b=b'$
\end{itemize}

Wiederhole dies 2x (akzeptiert ``wirklich'' nur falls beide Male akzeptiert wurde).

Dies ist korrektes Protokoll für $\overline{GI}$.

\end{definition}









\datum{25.01.2016}


% wiederholung: TODO merge with above

\begin{definition}
    \definiere{Interaktives Beweissysteme}
    \begin{align*}
        L \in \mathit{IP} & \text{ genau dann, wenn }
        \\
        & \exists \text{ polynomialzeitbeschränkte, probabilistische Turingmaschine Verifier } V:
        \\
        & \exists \text{ Prover } P:
        \\
        & x \in L \Rightarrow (P,V) \text{ akzepiert auf } x \text{ mit Wahrscheinlichkeit } \geq 75\%
        \\
        & x \notin L \Rightarrow \forall P: (P,V) \text{ verwirft auf } x \text{ mit Wahrscheinlichkeit } \geq 75\%
    \end{align*}

    Der Prover ist eine deterministische Turingmaschine ohne
    Ressourcenbeschränkung. In jeder Runde schickt der Verifizierer eine Frage
    an den Prover und der Prover eine Antwort an den Verifizierer.
    Die Anzahl der Runden ist polynomiell beschränkt.
    Ausschließlich $V$ entscheidet, ob akzeptiert wird.

\end{definition}




 % so jetzt geht's los


\subsection{Graphisomorphismus}
\label{sec:graphisomorphismus}


Verifizierer für das Problem, ob zwei Graphen \underline{nicht} isomorph sind, $\overline{GI}$:

Gegeben: $G_1 = (V_1, E_1), G_2 = (V_2, E_2)$

Falls $|V_1| \neq |V_2|$, dann verwerfen.

Wähle zufällig $i \in \{1,2\}$ und Permutation $\phi$ auf $V_1$.
Übermittle $G_3 = \phi(G_i)$ an den Prover.

Prover antwortet mit einer Zahl $j \in \{1,2\}$. Falls $i=j$, dann akzeptieren, sonst verwerfen.

Wenn $G_1 \not\cong G_2$, so kann der intendierte Prover so antworten, dass
$V$ mit Sicherheit akzepiert.
\\
Wenn $G_1 \cong G_2$: Falls $G_3$ übermittelt wird und $P$ richtig rät, so ist
das in 50\% der Fälle die falsche Antwort, da ja
$G_3 = \phi(G_1) = \phi(\psi^{-1}(G_2))$
wenn $\phi_1: G_1 \rightarrow G_2$ Isomorphismus
\\
$\Rightarrow$ Akzeptanzwahrscheinlichkeit $\leq$ 50\%.



% TODO tex these with tikz

%\begin{beispiel}

    %G_1      1 - 2
             %| / |
             %3 - 4

    %G_2      1   2
             %| X |
             %3 - 4

       %sind isopmorph

%\end{beispiel}


%\begin{beispiel}

    %G_1      1 - 2
             %| \
             %3   4

    %G_2      1 - 2
             %| /
             %3   4

       %sind nicht isopmorph

%\end{beispiel}





\subsection{Unerfüllbarkeit von 3KNF}
\label{sec:unerfullbarkeit_von_3knf}


Ziel: $\mathit{co-NP} \subseteq \mathit{IP}$

Konstruktion eines IP-Protokolls für Unerfüllbarkeit von 3KNF.

Vorgelegt sei eine 3KNF $\phi$.
\\
Gefragt ist $\phi(\eta) = 0$ für alle Belegungen $\eta$.
\\
$\eta: Vars(\phi) \rightarrow \{0.1\}$


Lösung ist die Arithmetisierung.
Man ordnet jeder KNF $\phi$ ein Polynom $[\phi]$ in denselben Variablen zu.
\begin{itemize}
    \item $[ x ] = x$
    \item $[ \phi \land \psi ] =  [\phi] * [\psi]$
    \item $[ \phi \lor \psi ] =  1 - (1 - [\phi]) * (1 - [\psi])$
    \item $[ \neg \phi ] =  1 - [\phi]$
\end{itemize}

Falls $\eta: Vars \rightarrow \{0,1\}$ \\
$\phi [\eta] = [\phi](\eta)$ \\
Durch Induktion über Aufbau von $\phi$.

$\phi = (x \lor \neg y) \land z$ \\
$[\phi] = (1 - (1-x) (1-(1-y))) * z$


Wenn $\phi$ eine 3KNF mit $n$ Variablen und $m$ Klauseln ist, dann ist $[\phi]$
ein Polynom in $n$ Variablen (denselben) und vom Grad $3m$.





Wir erweitern die Syntax von Polynomen um den Operator
$SUM_x$ wobei $SUM_x(p) = p(x=0) + p(x=1)$.

Jetzt ist $\phi$ unterfüllbar (mit Variablen $x_1 \dots x_2$) genau dann, wenn
\\
$SUM_{x_1}(SUM_{x_2}(SUM_{x_3}(\dots  SUM_{x_n}([\phi]) )))$.

Wir entwerfen ein IP-Protokoll für das Problem $p(a_1 \dots a_n) = b$?,
wobei $p$ verallgemeinertes Polynom ist.

Alle Zwischenergebnisse bei der Auswertung eines erweiterten Polynoms
sind durch $2^m$ beschränkt (und größer oder gleich 0).\\
Wir können daher alle Berechnungen in $\Z / p\Z (\Z_p), (\mod p)$ durchführen
(statt in $\R$), sofern $p$ eine Primzahl $\geq 2^m$ ist.

Ganz zu Beginn einigen sich $P$ und $V$ also auf solch eine Primzahl $p$.
Anschließend finden alle Berechnungen $\mod p$ statt.
Das Protokoll läuft über so viele Runden, wie das verallgemeinerte Polynom $p$ groß ist.


Falls $p = x_i$: Verifizierer prüft, ob $a_i = b$. Wenn ja, akzeptiere sonst verwerfen.
\\
Falls $p = p_1 * p_2$: Verifizierer fragt Prover nach Zwischenergebnissen
$b_1, b_2$ und prüft, ob $b_1 * b_2 = b$ und in folgenden Runden wird dann
bestätigt, dass $p_1(a) = b_1, p_2(a) = b_2$.



Fülle $p = p_1 + p_2$, $p = P_1 - P_2$ \\
$p = 1$ analog.

Fall $p = SUM_x p_1(x_1 \dots x_n, x)$
\\
Gefragt ist $p_1(a_1 \dots a_n, 0) + p_1(a_1 \dots a_n, 1) = b$.
\\
Verifizierer bittet Prover um die Koeffizienten (in $\Z_p$) es Polynoms
(vollständig ausmultipliziert.)
\\
$q(x) = p_1(a_1 \dots a_n, x)$ Polynom in 1 Variablen vom Grad $3m$.
\\
Verifizierer prüft, ob $q(0) + q(1) = b$.
\\
Verifizierer wählt zufällig $a \in \Z/p\Z$.
\\
In weiteren Runden wird bestätigt, dass
$p_1(a_1 \dots a_n, a) = q(a)$

Das funktioniert intuitiv deshalb weil bei Polynomen eine zufällige Einsetzung
fast so gut wie eine symbolische Variablenbelegung ist.










\datum{28.01.16}


\textbf{IP-Protokoll:}

Zu Beginn jeder Runde liegt ein verallgemeinertes Polynom $p$ in Variablen $x_1
\dots x_n$ vor und Werte $a_1 \dots a_k \in \Z_p$, wobei $p$ eine Primzahl $\geq
2^n$ ist.

$b \in \Z_p$
\\
Es gibt dann zu bestätigen, dass $p(a_1 \dots a_k) = b$.
\\
Gestartet wird mit
% todo ? über =
$SUM_{x_1} \dots SUM_{x_n}[q] =? 0$ für $k = 0$.
\\
Ist $p(a_1 \dots a_k) \equiv SUM_x p_1(a_1 \dots a_k, x)$
\\
* Wählt $V a \in \Z_p$ zufällig und lässt sich das Polynom
$q_(x) = p_1(a_1 \dots a_k, x)$
von P geben.
\\
$V$ prüft, ob $q(0) + q(1) = b$ und in der nächsten Runde wird bestätigt, dass
$p_1(a_1 \dots a_k, a) = q(a)$
\\
Falls $\phi$ unerfüllbar ist, dann kann P alle Fragen wie gewünscht beantworten und
$V$ wird zu 100\% überzeugt.
Falls $phi$ erfüllbar ist, ist die einzige Möglichkeit, dass P den Verifier
fälschlicherweise überzeugt, dass ein oder mehrere Male ander Stelle * zufällig
eine der $\leq 3m$ Nullstellen von $p_1(a, x) - q(x)$ wobei $p_1(a, x) \neq
q(x)$, erwischt wurde.

Die Wahrscheinlichkeit, dass das nicht passiert, ist $\geq (1 -
\frac{3m}{p})^2$. (Der Bruch bezeichnet die Größe des Körpers $\Z_p$)
\\
$p \geq c^n \leadsto \text{ Wahrscheinlichkeit } \geq 75\%$ wie gewünscht.

%TODO begründung dazu kommt am montag





\textbf{Verallgemeinerung:}


$\mathit{IP} \supseteq \mathit{PSPACE}$

$\mathit{QBF} \in \mathit{IP}$

Verallgemeinerte Polynome werden um zusätzliche Operatoren erweitert:
\begin{itemize}
    \item $\mathit{EX}_x p(x) = 1 - (1 - p(0))(1 - p(1))$
    \item $\mathit{ALL}_x p(x) = p(0) * p(1)$
    \item $\mathit{RED}_x p(x) = p(0) + x * (p(1) - p(0))$
\end{itemize}
$\mathit{RED}_x p(x)$ stimmt mit $p(x)$ auf $[0,1)$ überein.
$\mathit{RED}_x p(x)$ hat in $x$ aber Grad 1.

Einer QBF-Formel $\phi$ kann man mit diesen Operatoren ein verallgemeinertes
Polynom $[\phi]$ vom Grad $n$ (Anzahl der Variablen in $\phi$) zuordnen, sodass
$[\phi] = 1 \Leftrightarrow \phi \text{ wahr (keine freien Variablen)}$.




\begin{satz}

Zusammengefasst gilt:
$\mathit{IP} = \mathit{PSPACE}$

Es gibt aber Orakel $A$, sodass
$\mathit{IP}^A \neq \mathit{PSPACE}^A$

\end{satz}








\subsection{Probabilistically Checkable Proof}
\label{sec:probabilistically_checkable_proof}


\begin{definition}
    Bei \definiere{Probabilistically Checkable Proof} (PCP) gibt es einen
    Verifier wie bei IP.
    Statt eines interaktiven Provers gibt es jetzt einen ``Beweis'' $\Pi \in
    \{0,1\}^N$.
    $V$ stellt Fragen an $\Pi$ in Form eines Bitstrings $q \in \{0,1\}^{\log
    N}$.
    Diese werden mit dem $q.$ Bit von $\Pi$ beantwortet.

    \begin{align*}
        L \in \mathit{PHP} &\\
        \Leftrightarrow \exists V . & \\
                                    & x \in L \Rightarrow \exists \Pi . (\Pi, V) \text{ akzepiert mit
    Wahrscheinlichkeit } \geq 75\% \\
                                    & x \notin L \Rightarrow \forall \Pi . (\Pi, V) \text{ akzepiert mit
    Wahrscheinlichkeit } \leq 25\%
    \end{align*}

\end{definition}

Falls Rechenzeit von $V$ $\leq p(n)$, wobei $n = |x|$ und $p$ Polynom,
dann machen nur Beweise $\Pi$ mit $|\Pi| \leq 2^{p(n)}$ Sinn.

Offensichtlich ist $\mathit{IP} \subseteq \mathit{PCP}$.
Die Umkehrung ist fraglich, denn $\mathit{PCP} = \ComplexityClassNEXP$. Warum
ist nicht offensichtlich $\mathit{PCP} \subseteq \mathit{IP}$.

Einschränkungen:
\begin{itemize}
    \item $\mathit{PCP}(\psi(n), q(n)) = \{ L | \exists \mathit{PCP}
            \text{ Verifier } V $, der bei Eingabe $x$ nur $\bigO(\psi(|x|))$
            Zufallsbits verwendet und $\bigO(q(|x|))$ Anfragen stellt.
            $\}$
\end{itemize}



\begin{satz}
    \textbf{PCP-Theorem}

    $\ComplexityClassNP = \mathit{PCP}(\log(n), 1)$

    $c * \log(n)$ Zufallsbits
    \\
    $d$ Fragen
    \\
    Insgesamt $2^{c * \log(n)} * d = n^c * d = poly(n)$

    Der Beweis hat polynomielle Größe, kann also geraten werden.
    \\
    $\Rightarrow P(P(\log(n), 1)) \subseteq \ComplexityClassNP$

\end{satz}


Im Buch ist ein Beweis von
$\ComplexityClassNP \subseteq \mathit{PCP}(poly(n), 1)$








\textbf{ Problem MAX-CLIQUE-APPROX }

Gegeben ein Graph $G = (V, E)$.

Man berechne eine Clique in $G$, die mindestens halb so groß ist, wie die
maximale Clique.
Das Problem nennt man MAX-CLIQUE-APPROX (MCA).


\begin{satz}
    Falls $\ComplexityClassP \neq \ComplexityClassNP$, dann gibt es keinen
    polynomiellen Algorthmus für das Problem MAX-CLIQUE-APPROX.
\end{satz}

\begin{beweis}
    Wir geben uns einen $\mathit{PCP}(\log(n), 1)$ Verifier $V$ für SAT vor.
    Außerdem einen hypothetischen Polynomialzeit-Algorithmus für MCA.

    Daraus bauen wir ein Polynomialzeit-Entscheidungsverfahren für SAT.

    Sei 3KNF $x$ vorgelegt.
    Es werden $d = \bigO(\log(|x|))$ Zufallsbits verwendet und $t = \bigO(1)$
    Fragen gestellt.

    Transcript:
    $(r, q_1, a_1, q_2, a_2, \dots q_t, a_t)$,
    wobei $r$ die Zufallsbits und $q_i$ die Fragen sind.
    Dieses Transcript beschreibt einen möglichen Lauf der Berechnung von $V$.

    Nachdem $V$ eine Turingmaschine ist, hängen die $q_i$ nur ab von $r$ und
    $a_1 \dots a_t$.
    Gültige Transcripts sind nur solche, die tatsächlich durch $V$ in
    Erscheinung treten.
    Die Anzahl der Transcripts ist $2^d * 2^t = poly(n)$.

    Bilde Graphen $G = (V, E)$, wobei $V$ die Menge der Transcripts und $E$ die
    Konsistenz der Transcripts ist.

    $(r, q_1, a_1, q_2, a_2, \dots q_t, a_t)$ und
    $(r', q_1', a_1', q_2', a_2', \dots q_t', a_t')$
    sind konsistent, wenn $q_i = q_i' \Rightarrow a_i = a_i'$.

    Jeder funktionierende Beweis $\Pi$ liefert eine Clique der Größe $\geq 0.75
    * 2^d$, wobei $d$ so in etwa $\bigO(\log(i))$ ist oder so.
    \\
    Umgekehrt gibt es keinen Beweis mit Akzeptanzwahrscheinlichkeit $\geq 25\%$,
    dann gibt es keine Clique $\geq 0.25 * 2^d$.

    Mit dem hypothetischen Algorthmus für MCA berechnet man eine Clique, die
    mindestens halb so groß wie die größte Clique ist.
    Falls $> 0.5 * 0.75 * 2^d = 0.375 * 2^d$, dann akzeptieren, sonst verwerfen.

\end{beweis}






\textbf{Schaltkreiskomplexität}

Schaltkreise aus UND-, ODER-Gattern. NICHT-Gatter stellen

XOR mit NAND gebaut:
\url{https://www.circuitlab.com/circuit/5m6zgn/xor-gate-with-nand-gates/}


UND-Gatter gefolgt von UND können verschmolzen werden. ODER-Gatter auch. (Also
mehr Eingänge statt hintereinander stecken.)

Daher können wir Schaltkreise als geschichtet annehmen, wobei jede Schicht
Gatter derselben Art enthält und UND, ODER abwechseln.

Die Größe eines Schaltkreises ist die Anzahl der Gatter.
\\
Die Tiefe eines Schaltkreises ist die Anzahl der Schichten.


DNF:\\
Tiefe 3 genügt, wenn man exponentielle Größe in Kauf nimmt.


Satz: First-Save-Sipser oder so:\\
Paritätsfunktion ($\Sigma \mod 2$) kann nicht mit polynomieller Größe und
konstanter Tiefe berechnet werden.
\\
(Tiefe $\log n$ leicht: Binärbaum aus XOR).
















% chapter probabilistische_algorithmen (end)

%%% Local Variables:
%%% mode: latex
%%% TeX-master: "0-main"
%%% ispell-local-dictionary: "de-neu"
%%% eval: (highlight-parentheses-mode)
%%% End:



\end{document}
